\documentclass[12pt,pdftex,noinfoline]{imsart}

\RequirePackage[OT1]{fontenc}
\usepackage{mymathstyle}


\makeatletter
\let\c@author\relax
\makeatother
\usepackage[style=authoryear-comp,backend=bibtex]{biblatex}
\addbibresource{references.bib}

\def\given{\,|\,}
\def\P{\mathbb{P}}
\def\E{\mathbb{E}}
\def\reals{\mathbb{R}}
\let\what\widehat
\let\tilde\widetilde
\let\phi\varphi

\renewcommand{\argmin}{\mathop{\rm argmin}}
\newcommand{\ess}{\mathop{\rm ess}}
\renewcommand{\argmax}{\mathop{\rm argmax}}

\begin{document}
\def\snote#1{${}^{#1}$}
\setlength{\parskip}{0.5em}
\begin{frontmatter}
{\bf\Large Universal approximation of relation functions\\[5pt] by inner products of neural networks}
%\affil[**]{Department of Statistics and Data Science, Yale University}
\begin{aug}
\vskip15pt
\address{
\begin{tabular}{ccccc}
{\normalsize\rm\bfseries Awni Altabaa}\snote{1} & {\normalsize\rm\bfseries John Lafferty}\snote{2}\\[5pt]
\end{tabular}
\vskip5pt
\footnotetext{
\snote{1}Department of Statistics and Data Science, Yale University; awni.altabaa@yale.edu.
\snote{2}Department of Statistics and Data Science, Wu Tsai Institute, Institute for Foundations of Data Science, Yale University; john.lafferty@yale.edu.
}
\today
\vskip10pt
}
\begin{abstract}
    Inner products of neural network feature maps arises in a wide variety of machine learning frameworks as a method of modeling relations between inputs.
    This work studies the approximation properties of inner products of neural networks.
    It is shown that the inner product of a multi-layer perceptron with itself is a universal approximator for symmetric positive-definite relation functions. In the case of asymmetric relation functions, it is shown that the inner product of two different multi-layer perceptrons is a universal approximator.
    In both cases, a bound is obtained on the number of neurons required to achieve a given accuracy of approximation. In the symmetric case, the function class can be identified with kernels of reproducing kernel Hilbert spaces, whereas in the asymmetric case the function class can be identified with kernels of reproducing kernel Banach spaces.
    Finally, these approximation results are applied to analyzing the attention mechanism underlying Transformers, showing that any retrieval mechanism defined by an abstract preorder can be approximated by attention through its inner product relations.
    This result uses the Debreu representation theorem in economics to represent preference relations in terms of utility functions.
\end{abstract}

\end{aug}
\end{frontmatter}

\section{Introduction}\label{sec:intro}

Machine learning systems must be able to represent and reason about relations between objects, either explicitly or implicitly. For example, a natural language understanding system takes a sequence of words as input and extracts information about the meaning of the words based on relations between them in the local context. Similarly, a scene analysis system considers the relations between the components of a scene in order to identify and interpret the objects. 

A common way to represent relations between objects is through inner products between feature representations of the form $\iprod{\phi(x)}{\psi(y)}$, where $x, y \in \calX$ are two objects and $\phi, \psi$ are neural network feature maps. Inner products posses properties which make them useful measures of similarity. The aim of this paper is to understand the representational power of this model by characterizing the class of relation functions $r: \calX \times \calX \to \reals$ which can be represented as inner products of neural networks.

Inner products between neural network feature maps of objects appear in many machine learning architectures. A notable example is the attention mechanisms that lie at the heart of sequence models like the Transformer~\parencite{vaswani2017attention}. In the Transformer, self-attention is implemented as
\begin{equation*}
    \begin{split}
        \alpha_{ij} &\gets \mathrm{Softmax}\paren{\bra{{\iprod{\phi_q(x_i)}{\phi_k(x_j)}}}_{j \in [n]}}_j\\
        x_i' &\gets \sum_{j=1}^{n} \alpha_{ij} \phi_v(x_j)
    \end{split}
\end{equation*}
where $\phi_q, \phi_k$, and $\phi_v$ are learned transformations and $\iprod{\phi_q(x_i)}{\phi_k(x_j)}$ represents a relation between $x_i$ and $x_j$, which determines how much $i$ should attend to $j$.

The Transformer models relations implicitly through its attention mechanism. Modeling relations through inner products of features is also central to many ``explicitly relational'' neural architectures~\parencite[e.g.,][]{webbEmergentSymbols2021,kergNeuralArchitecture2022,altabaaAbstractorsTransformer2023,altabaaRelationalConvolutionalNetworks2023}. For example, in the CoRelNet model~\parencite{kergNeuralArchitecture2022}, a similarity matrix is computed consisting of symmetric inner products between each pair of objects, $R_{i,\cdot} = \mathrm{Softmax}\pparen{\bra{\iprod{\phi(x_i)}{\phi(x_j)}}_{j\in[n]}}$.~\cite{altabaaRelationalConvolutionalNetworks2023} propose a Transformer-based architecture imbued with relational inductive biases by replacing the values $\phi_v(x_i)$ with vector representations which identify objects but are independent of object-level features.~\cite{altabaaRelationalConvolutionalNetworks2023} propose a relational architecture where the central operation is a type of ``convolution'' operating on a tensor of relations computed via inner products of feature maps.

\aanote[pdfcnote, noinline]{are we talking too much about our own work? what other work can we mention that uses inner products to model some kind of relation or similarity? talk more about attention?}

\aawarning[pdfcnote, noinline]{These two paragraphs are talking about architectures which involve inner products of neural network feature maps as a means to model relations? How do we want to organize this? a separate ``related work'' section or just right here? what other architectures do we want to mention? any other relevant well-known work? can mention relational Relational recurrent neural networks}

Siamese networks are another domain where understanding the function classes of inner products of neural network transformations is relevant~\parencite[e.g.,][]{chopraLearningSimilarityMetric2005}. There, the relevant quantity is a distance between transformations of two objects. If the distance is the Euclidean distance, then $\twonorm{\phi(x) - \phi(y)}^2 = \twonorm{\phi(x)}^2 + \twonorm{\phi(y)}^2 - 2 \iprod{\phi(x)}{\phi(y)}$, where again the inner product of neural networks arises.

\aawarning[pdfcnote, noinline]{[todo]: expand a bit on siamese networks. what were they designed to do? compare things...}

% universal approximation of feedforward neural networks has been extensively studied (e.g., classical references are ... Barron, Cybenko, etc.)
In this paper we characterize the function class of inner products of neural networks, showing that inner products of neural networks are universal approximators for relation functions. In particular, when the inner product of neural networks is symmetric (i.e., $\phi=\psi$), the function class is the class of continuous positive definite kernels. When the inner product is asymmetric (i.e., $\phi \neq \psi$), the function class is all continuous bivariate functions on $\calX \times \calX$. The symmetric case is related to kernels of reproducing kernel Hilbert spaces whereas the asymmetric case admits a connection to kernels of reproducing kernel Banach spaces. We apply these results to analyzing the attention operation, showing that any retrieval function described by arbitrary pre-order define 
\section{Function class of symmetric inner product relations}\label{sec:symmetric_relations}

Consider a symmetric relation $r: \calX \times \calX \to \reals$ modeled as the inner product of neural network encodings of a pair of inputs, $x, y \in \calX$,
\begin{equation}\label{eq:symmetric_iprod_relation}
    r(x, y) = \iprod{\phi(x)}{\phi(y)},
\end{equation}

\noindent where $\phi$ is a neural network. Symmetric relation functions modeled in this way are natural \textit{measures of similarity}. Intuitively, $\phi$ extracts features of the objects $x,y$, and the inner product computes a measure of similarity between those features. To see this more formally, suppose we have a normalized inner product relation (i.e., $\phi: \calX \to \bbS^{d}$), then we have
\begin{equation}
    \twonorm{\phi(x) - \phi(y)}^2 = 2 - 2 r(x, y).
\end{equation}

The Cauchy-Schwartz inequality states that
\begin{equation}
    r(x, y)^2 \leq r(x, x) r(y, y) = 1, \ \forall x, y \in \calX.
\end{equation}

Thus, $r(x,y)$ is large and close to $1$ when $x$ and $y$ are similar (with respect to $\phi$), and close to $0$ when $x, y$ are dissimilar. Moreover, $r: \calX \times \calX \to \reals$ induces a geometry on $\calX$ with well-defined notions of distance ($d(x,y) = \sqrt{2 - 2 r(x,y)}$), angles ($\cos(\theta) = r(x,y)/\sqrt{r(x,x) r(y,y)}$), and orthogonality ($x \perp y \iff r(x,y) = 0$).

In this section, we characterize the class of relation functions which can be modeled by symmetric inner products of neural networks. We show that any continuous symmetriic positive definite kernel (i.e., Mercer kernel) can be approximated by a symmetric inner product of neural networks. This characterization is strict since $r(x,y) = \iprod{\phi(x)}{\phi(y)}$ is a Mercer kernel for any $\phi$ because $\forall g \in L^2(\calX)$ we have,
\begin{equation*}
    \begin{split}
        \int \int g(x) r(x,y) g(y) dx dy &= \int \int g(x) \iprod{\phi(x)}{\phi(y)} g(y) dx dy \\
        &= \iprod{\int g(x) \phi(x) dx}{\int g(y) \phi(y) dy} \\
        &= \twonorm{\int g(x) \phi(x) dx} \geq 0.
    \end{split}
\end{equation*}

% COMMENT [AWNI]: is this ^ unnecessary detail? could just say this, maybe there's no need for an argument?

The following result states that if $\phi$ is a universal approximator such as a multi-layer perceptron, for any positive-definite symmetric kernel $r$, there exists a neural network $\phi$ such that the induced inner product relation approximates $r$ arbitrarily well. To state the result, we begin with some definitions.

Consider a symmetric positive definite kernel $K: \calX \times \calX \to \reals$ on a compact euclidean space $\calX$. By Mercer's theorem~\parencite{mercerFunctionsPositive1909, sunMercerTheorem2005, micchelliUniversalKernels2006}, there exists non-negative eigenvalues $\set{\lambda_i}_{i=1}^{\infty}$ and continuous eigenfunctions $\set{\psi_i}_{i=1}^{\infty}$ such that for any $x,y \in \calX$, $K(x, y) = \sum_{i=1}^\infty \lambda_i \psi_i(x) \psi_i(y)$, and the convergence of the series is absolute and uniform over $\calX$. The following definition characterizes, in a sense, the complexity of a symmetric positive-definite kernel in terms its eigenfunction decomposition.

\begin{definition}[Kernel Spectrum Decay]\label{def:sym_pd_ker_specturm_decay}
	Let $K: \calX \times \calX \to \reals$ be a symmetric positive-definite kernel, with eigenvalues $\{\lambda_i\}_{i=1}^{\infty}$ and continuous eigenfunctions $\{\psi_i\}_{i=1}^{\infty}$. For $\epsilon > 0$, denote by $d_K(\epsilon) \in \naturals$ the minimum integer such that,
	\begin{equation*}
		\sup_{x, y \in \calX} \abs{K(x, y) - \sum_{i=1}^{d(\epsilon)} \lambda_i \psi_i(x) \psi_i(y)} \leq \epsilon.
	\end{equation*}
	Moreover, let $C(K, d) \coloneq \max_{i \in [d]} \max_{x \in \calX} \abs{\sqrt{\lambda_i} \psi_i(x)}$, and let $L(K, d) \coloneq \max_{i \in [d]} L_i$, where $L_i$ is the Lipschitz constant of $\sqrt{\lambda_i} \psi_i$.
\end{definition}

The quantity $d_K(\epsilon)$ associated with a kernel $K$ describes the number of eigenfunctions needed to describe $K$ up to an error of $\epsilon$. For any symmetric positive-definite kernel $K$, $d_K(\epsilon)$ is finite for any $\epsilon > 0$. Similarly, the quantity $L(K, d)$ describes the level of ``regularity'' of the first $d$ eigenfunctions of $r$. Note that $C(K, d)$ is finite for any $d$ since $\calX$ is assumed compact and the eigenfunctions $\psi_i$ are continuous. % Moreover, $d_K(\epsilon)$ increases as $\epsilon \to 0^+$, and $C(K, d), L(K, d)$ are increasing in $d$. Hence, $C(K, d_K(\epsilon)), L(K, d_K(\epsilon))$ is decreasing in $\epsilon$.

Next, to state our result, we define notation which describes the efficiency of a universal function approximator on a space $\calX$.
\begin{definition}[Efficiency of function approximator]\label{def:univ_approx_efficiency}
	Consider a class of neural networks on a space $\calX$ with a scalable parameter space (e.g., scalable width, depth, number of neurons, etc). Let $\Theta(n)$ be the parameter space at scale $n$. Let $\calF$ be a class of functions on $\calX$. For $\epsilon > 0$, denote by $\calN_\calF(\epsilon) \in \naturals$ the minimum integer such that for any function $f \in \calF$, there exists a neural network $\theta \in \Theta(\calN_\calF(\epsilon))$ that approximates $f$ uniformly on $\calX$ with error bounded by $\epsilon$. That is,
	\begin{equation*}
		\sup_{f \in \calF} \, \min_{\theta \in \Theta(\calN(\epsilon))} \,  \sup_{x \in \calX} \abs{f(x) - g_\theta(x)} \leq \epsilon
	\end{equation*}
\end{definition}

Note that we've considered approximation in the $L^\infty$ norm in this definition. We may also consider alternative norms. There exists an extensive literature which investigates the properties of $\calN_\calF(\varepsilon)$ for different function classes $\calF$ (e.g., smoothness assumptions), different neural network classes (e.g., shallow vs deep), and alternative norms (e.g., $L^p$). Classical results include~\parencite{cybenkoApproximationSuperpositions1989,barronUniversalApproximation1993,hornikMultilayerFeedforward1989}.

We are now ready to formally state our result.

%To-Do: define \calF in theorem statement below. \calF = cts functions X to reals; or L(r, d_r(eps/2))-lipschitz functions from X to reals...
\begin{theorem}[Function class of symmetric inner product relational neural networks]\label{theorem:symmetric_inner_prod_rels_func_class}
	Suppose the data lies in a compact euclidean space $\mathcal{X}$. Consider the relation model,
	\begin{equation*}
		\hat{r}(x, y) := \iprod{\phi_{\theta}(x)}{\phi_{\theta}(y)},
	\end{equation*}
	where $\phi_{\theta}: \calX \to \reals^d$ is a multi-layer perceptron with parameters $\theta$. % Let $\calF$ be the space of $L(r, d_r(\epsilon/2))$-Lipschitz functions from $\calX$ to $\reals$.
	Then, for any symmetric positive-definite kernel $r: \calX \times \calX \to \reals$, there exists a multi-layer perceptron with parameters $\theta$ such that $\hat{r}$ approximates $r$ uniformly over $(x,y) \in \mathcal{X}\times\mathcal{X}$. More precisely, for all $\epsilon > 0$, there exists a neural network with at most $d_r(\epsilon / 2) \cdot \calN_\calF\paren{\frac{\epsilon}{4 C(r) d_r(\epsilon / 2)}}$ neurons and parameters $\theta$ such that
    \[\sup_{x,y \in \mathcal{X}}{\abs{r(x,y) - \hat{r}(x,y)}} < \epsilon.\]
	Here, $\calF$ is the space of $L(r, d_r(\epsilon/2))$-Lipschitz functions from $\calX$ to $\reals$.
\end{theorem}

\begin{proof}
	By Mercer's theorem~\parencite{mercerFunctionsPositive1909, sunMercerTheorem2005, micchelliUniversalKernels2006}, there exists $(\psi_i)_{i \in \mathbb{N}}$, $\lambda_i \geq 0$ such that $r(x,y) = \sum_{i=1}^{\infty}{\lambda_i \psi_i(x) \psi_i(y)}$, where $\psi_i$ and $\lambda_i$ are eigenfunctions and eigenvalues of the integral operator
	\begin{align*}
		T_r&: L^2(\mathcal{X}) \to L^2(\mathcal{X}) \\
		T_r(f) &= \int_{\mathcal{X}}{r(\cdot, x) f(x) dx}.
	\end{align*}
	Furthermore, the convergence of the series is uniform:
	\begin{equation}
		\lim_{n \to \infty} \sup_{x,y \in \mathcal{X}} \lvert r(x,y) - \sum_{i=1}^{n}{\lambda_i \psi_i(x) \psi_i(y) \rvert} = 0
	\end{equation}
	Using the notation of~\Cref{def:sym_pd_ker_specturm_decay}, we have that
	\begin{equation}\label{eq:proof_mercer_thm_unif_abs_cv}
		\sup_{x,y \in \mathcal{X}} \left\lvert r(x,y) - \sum_{i=1}^{d_r(\epsilon/2)}{\lambda_i \psi_i(x) \psi_i(y)} \right\rvert < \frac{\epsilon}{2},
	\end{equation}
	where $d_r(\epsilon/2)$ is defined as in~\Cref{def:sym_pd_ker_specturm_decay}.

	Let $d \coloneqq d_r(\epsilon / 2)$ be the output dimension of the neural network $\phi_\theta$ such that it maps from $\calX$ to $\reals^d$. The function to be approximated by $\phi_\theta$ is $(\sqrt{\lambda_1} \psi_1, \ldots, \sqrt{\lambda_{d}} \psi_{d})$, the first $d$ eigenfunctions of the kernel relation $r$. By the universal approximation property of multi-layer perceptrons, for any $\epsilon_1 > 0$, there exists a neural network with parameters $\theta$ such that
	\begin{equation}\label{eq:proof_NN_UAP}
		\sup_{x\in \mathcal{X}}{\abs{(\phi_\theta(x))_i - \sqrt{\lambda_i} \psi_i(x)}} < \epsilon_1, \ \forall i \in \{1, \ldots, d\},
	\end{equation}
	where $(\phi_\theta(x))_i$ is the $i$th component of $\phi_\theta(x)$. Moreover, using the notation defined in~\Cref{def:univ_approx_efficiency}, there exists such a neural network with at most $d \cdot \calN_\calF(\epsilon_1)$ neurons, where $\calF \coloneq \mathrm{Lip}(\calX, \reals; L(r, d_r(\epsilon/2)))$ is the set of $L(r, d_r(\epsilon/2))$-Lipschitz functions from $\calX$ to the real line. %Some classical results on universal approximation of neural networks are \parencite{hornikMultilayerFeedforward1989, cybenkoApproximationSuperpositions1989, barronUniversalApproximation1993}, which characterize $\calN_\calF(\epsilon_1)$. 

	Now note that the approximation error for $r$ is bounded by
	\begin{equation}\label{eq:proof_approx_bound}
		\begin{split}
			\sup_{x, y \in \mathcal{X}}&{
				\left\lvert r(x,y) - \langle \phi_\theta(x), \phi_\theta(y) \rangle \right\rvert}\\
			&= \sup_{x, y \in \mathcal{X}}{
				\left\lvert r(x,y) - \sum_{i=1}^{d}{(\phi_\theta(x))_i (\phi_\theta(y))_i} \right\rvert} \\
			&\leq \sup_{x,y \in \mathcal{X}}{ \left(
				\left\lvert r(x,y) - \sum_{i=1}^{d}{\lambda_i \psi_i(x) \psi_i(y)} \right\rvert
				+ \left\lvert \sum_{i=1}^{d}{\lambda_i \psi_i(x) \psi_i(y) - (\phi_\theta(x))_i (\phi_\theta(y))_i} \right\rvert  \right) }
		\end{split}
	\end{equation}
	The first term is bounded by $\frac{\epsilon}{2}$ by~\eqref{eq:proof_mercer_thm_unif_abs_cv}. The second term can be bounded uniformly on $x,y$ by
	\begin{equation*}
		\begin{split}
			&\left\lvert \left(\sum_{i=1}^{d}{\lambda_i \psi_i(x) \psi_i(y)}\right) - \langle \phi_\theta(x), \phi_\theta(y) \rangle \right\rvert  \\
			&\leq \sum_{i=1}^{d}{ \abs{ \lambda_i \psi_i(x) \psi_i(y) - (\phi_\theta(x))_i (\phi_\theta(y))_i}} \\
			&\leq \sum_{i=1}^{d}{\paren{
				\abs{\sqrt{\lambda_i} \psi_i(y)} \abs{\sqrt{\lambda_i} \psi_i(y) - (\phi_\theta(y))_i}
				+ \abs{\sqrt{\lambda_i} \psi_i(y)} \abs{\sqrt{\lambda_i} \psi_i(x) - (\phi_\theta(x))_i}
				}} \\
			&\stepa{\leq} C(r) \sum_{i=1}^{d}{\paren{
				\abs{\sqrt{\lambda_i} \psi_i(y) - (\phi_\theta(y))_i}
				+ \abs{\sqrt{\lambda_i} \psi_i(x) - (\phi_\theta(x))_i}
				}} \\
			&\stepb{\leq} 2 C(r) \cdot d \cdot \epsilon_1,
		\end{split}
	\end{equation*}
	where step (a) is by the definition $C(r) \coloneqq \max_{x \in \calX} \lvert \sqrt{\lambda_i} \psi_i(x) \rvert$ (\Cref{def:sym_pd_ker_specturm_decay}) and step (b) is by~\Cref{eq:proof_NN_UAP}.

	Let the neural network approximation error be $\epsilon_1 = \frac{\epsilon}{4 C(r) d_r(\epsilon / 2)}$ such that the above is bounded by $\epsilon / 2$. 

	Then, by~\eqref{eq:proof_approx_bound}, we have that
	\begin{equation*}
		\sup_{x, y \in \mathcal{X}}{
			\abs{r(x,y) - \iprod{\phi_\theta(x)}{\phi_\theta(y)}}} \leq \frac{\epsilon}{2} + \frac{\epsilon}{2} = \epsilon.
	\end{equation*}

	Hence, the relation function $r$ is approximated uniformly on $\calX \times \calX$ by an inner product of neural networks with at most $d_r(\epsilon / 2) \cdot \calN_\calF\paren{\frac{\epsilon}{4 C(r) d_r(\epsilon / 2)}}$ neurons.

\end{proof}

\Cref{theorem:symmetric_inner_prod_rels_func_class} states that pairwise relations modeled as inner products of neural networks can capture any symmetric positive definite kernel. Moreover, the scale of neural network needed to achieve a particular approximation error is characterized in terms of the complexity of the kernel relation function and the efficiency of the neural network function class. In particular, this dependence is expressed in the kernel relation's spectrum decay $d_r(\cdot)$ and the neural network's efficiency $\calN_\calF(\cdot)$.

The efficiency of neural network's in approximating arbitrary functions has been studied extensively. Different results can give different bounds on $\calN_\calF(\cdot)$. The following result follows from a particular bound obtained in~\parencite{poggioWhyWhenCan2017}.

\begin{corollary}\label{cor:sym_iprod_kernel_neuron_bound}
	Consider the setting of~\Cref{theorem:symmetric_inner_prod_rels_func_class} with the neural network $\phi_\theta$ a shallow neural network with ReLU activations,
	\begin{equation*}
		\phi_\theta(x) = \sum_{k=1}^{n} a_k \mathrm{ReLU}(\iprod{w_k}{x} + b_k),
	\end{equation*}
	where $a_k, b_k \in \reals$, $w_k \in \reals^{\mathrm{dim}(\calX)}$, and $\theta = (a_1, \ldots, a_n, b_1, \ldots, b_n, w_1, \ldots, w_n) \in \reals^{n (\mathrm{dim}(\calX) + 2)}$, and $n$ is the number of neurons. Then, there exists $\theta$ achieving $\sup_{x,y \in \mathcal{X}}{\abs{r(x,y) - \hat{r}(x,y)}} < \epsilon$ in~\Cref{theorem:symmetric_inner_prod_rels_func_class} with a number of neurons $n$ of the order
	\[\calO\paren{d_r(\epsilon / 2) \cdot \paren{\frac{\epsilon}{4 C(r) d_r(\epsilon / 2) L(r, d_r(\epsilon/2))}}^{-\mathrm{dim}(\calX)}}.\]
\end{corollary}
\begin{proof}
	This follows by~\Cref{theorem:symmetric_inner_prod_rels_func_class} and~\parencite[Theorem 4]{poggioWhyWhenCan2017}.
\end{proof}

% TODO: double check that I'm interpreting poggio's result correctly. Sorting through the literature of universal approximation results is a bit of a challenge: they make different kinds of regularity assumptions (e.g., smoothness up to a certain order, etc) and sometimes have distribution-dependent rather than uniform bounds.

Using the same analysis, we can obtain bounds for different regularity assumptions on the kernel eigenfunctions by using different universal approximation results. Similarly, we can consider approximation under different norms (e.g., under distribution-aware norms like $L^p$ norms). The above analysis also holds for universal approximators other than feedforward neural networks, provided that the appropriate universal approximation property holds.

\begin{remark}
	In the results above, the number of neurons is bounded by ``$d_r \cdot \calN$''. This is an upper bound assuming each of the $d_r$ kernel eigenfunctions is modeled independently. In practice, the size of the neural network needed would be smaller since the eigenfunctions can be approximated with a single MLP with a $d_r$-dimensional output, allowing for computations to be re-used across several output dimensions. %In fact, for a shallow neural network, the number of neurons would simply be $\calN_\calF(\epsilon(4 C(r) d_r(\epsilon / 2))^{-1})$---although the number of synapses at the output layer would of course be $d_r(\epsilon / 2) \cdot \calN_\calF(\epsilon(4 C(r) d_r(\epsilon / 2))^{-1})$.
\end{remark}

% TODO: check this remark. also, apply to corollary below? (this is a shallow network).
\section{Function class of asymmetric inner product relations}\label{sec:asymmetric_relations}

In the previous section we considered symmetric inner product relations where the encoder of the first object is the same as the encoder of the second object. When the underlying relation being modeled is a symmetric `similarity' relation, this is a useful inductive bias. However, in general, relations between objects can be asymmetric. For example, order relations are asymmetric (in fact, anti-symmetric). Such relations cannot be captured by symmetric inner products. In this section, we consider modeling a general (asymmetric) relation $r: \calX \times \calX$ as the inner product of two different neural network encodings of a pair of objects,
\begin{equation}\label{eq:asymmetric_iprod_relation}
    r(x, y) = \iprod{\phi(x)}{\psi(y)},
\end{equation}
where $\phi, \psi: \calX \to \reals^d$ are two neural networks.

In this section, we show that inner products of universal approximators (e.g., when $\phi, \psi$ are multi-layer perceptrons) can approximate any continuous function on $\calX \times \calX$.

We begin with the following simple lemma which states when the object space $\calX$ is finite, any relation function can be represented as the inner product between two encodings.

\begin{lemma}\label{lemma:finite_space_rel}
    Suppose $\calX$ is a finite space. Let $r: \calX \times \calX \to \reals$ be any relation function. Then, there exists $d \leq \abs{\calX}$ and $\phi, \psi: \calX \to \reals^{d}$ such that,
    \begin{equation*}
        r(x, y) = \iprod{\phi(x)}{\psi(y)}, \ \forall x, y \in \calX.
    \end{equation*}
\end{lemma}

\begin{proof}
    Let $x_1, \ldots, x_n$ be an enumeration of $\calX$ where $m = \abs{\calX}$. Let $R \in \reals^{n \times n}$ such that $R_{ij} = r(x_i, x_j)$. There exists many decompositions of the matrix $R$ which would induce valid encodings $\phi, \psi$. One example is rank decomposition. Let $d = \mathrm{rank}(R)$. Then, there exists matrices $P, Q \in \reals^{d \times n}$ such that $R = P^\top Q$. Let $\phi, \psi: \calX \to \reals^{d}$ be defined by
    \begin{equation}
        \phi(x_i) = P_{i, \cdot}, \ \psi(x_i) = Q_{\cdot, i}, \ \forall i \in [m].
    \end{equation}

    Then, $r(x, y) = \iprod{\phi(x)}{\psi(y)}$ for all $x, y \in \calX$.
\end{proof}

Note that if each $x \in \calX$ is a one-hot vector in $\reals^{\abs{\calX}}$, then the result above holds with linear maps $\phi, \psi$. Although very simple, this result has direct implications to domains such as language modeling where $\calX$ is a discrete set of tokens, and hence finite. In such cases,~\Cref{lemma:finite_space_rel} tells us that any relation function can be approximated by inner products of feature maps (i.e., of the form present in the attention mechanisms of Transformers). Moreover, in the case of language, there may be a low-rank structure (e.g., depending on syntax, semantics, etc.) enabling a more modest dimension of the feature maps, $d \ll \abs{\calX}$.

Next, we proceed to show that arbitrary continuous relation functions can be approximated by inner products of two different neural networks. Our strategy will be to first quantize the space $\calX$ then apply the construction above for the finite case.

\begin{theorem}\label{theorem:asymemtric_inner_prod_rel_func_class}
    Suppose the relation function $r: \calX \times \calX \to \reals$ is continuous. In particular, for any $\epsilon > 0$, there exists $\delta(\epsilon) > 0$ such that for any $x, y, \tilde{x}, \tilde{y}$ satisfying $\norm{x - \tilde{x}} \leq \delta$, $\norm{y - \tilde{y}} \leq \delta$, we have $\abs{r(x, y) - r(\tilde{x}, \tilde{y})} \leq \epsilon$. Then, for any approximation error $\epsilon > 0$ there exists multi-layer perceptrons $\phi, \psi$ such that
    \begin{equation*}
        \abs{r(x,y) - \iprod{\phi(x)}{\psi(y)}} \leq \epsilon, \quad \text{Lebesgue-almost everywhere.}
    \end{equation*}
    Moreover, $\phi, \psi$ can be constructed such that $\phi = L_\phi \circ \eta$, $\psi = L_\psi \circ \eta$ where $\eta$ is a shared 2-layer MLP with $N = \calO(\delta^{- 2 \,\dim(\calX)})$ neurons and $L_\phi, L_\psi$ are linear projections onto $n$-dimensional space, with $n = \calO(\delta^{- \dim(\calX)})$ and $\delta = \delta(\epsilon)$.
\end{theorem}
\begin{proof}
    Let $x_1, \ldots, x_n \in \calX$ be a set of points in $\calX$. Define the Voronoi partition by
    \[V^{(i)} = \sset{x \in \calX \,|\, \norm{x - x_i} \leq \norm{x - x_j} \ \forall j \neq i}.\]
    Let $x_1, \ldots, x_n$ be uniformly distributed in $\calX$ and $n = \calO(\delta(\epsilon)^{- \dim(\calX)})$ so that the maximal diameter of the sets $V^{(1)}, \ldots, V^{(n)}$ is bounded by $\delta(\epsilon)$, $\max_{i \in [n]} \mathrm{diam}(V^{(i)}) \leq \delta$. Let $q: \calX \to \sset{x_1, \ldots, x_n}$ be the quantizer which maps each $x$ to the closest element in $\sset{x_1, \ldots, x_n}$.

    \citet{wuExplicitNeuralNetwork2018} explicitly construct a two-layer neural network $\eta:\calX \to \{0, 1\}^{n}$ such that
    \[{(\eta(x))}_i = 1 \iff x \in V^{(i)}.\]
    The construction contains $n (n - 1)$ neurons in the first layer and $n$ neurons in the second layer, both with the threshold function $\sigma(x) = \bm{1}\sset{x \geq 0}$ as the activation function. Note that sigmoidal activation functions can approximate the step function arbitrarily well. The weights between the first and second layer are sparse. The neural network takes the form,
    \begin{equation*}
        \begin{aligned}
            z_{k,j}^{(1)}(x) &= \sigma\paren{w_{k,j}^{(1)} \cdot x - b_{k,j}^{(1)}}, \quad k,j \in [n], k \neq j\\
            z_k^{(2)}(x) &= \sigma\paren{w_k^{(1)} \cdot \bm{z}^{(1)}(x) - b^{(2)}}, \quad k \in [n] \\
            \eta(x) &= \bm{z}^{(2)}(x) = \paren{z_1^{(2)}(x), \ldots, z_n^{(2)}(x)},
        \end{aligned}
    \end{equation*}
    where the weights are $w_{k,j}^{(1)} = x_k - x_j, b_{k,j}^{(1)} = \frac{1}{2} \iiprod{x_k - x_j}{x_k + x_j}, (w_{k}^{(2)})_{a,b} = \bm{1}\sset{a = k}, b^{(2)} = n - 1$, and $\bm{z}^{(1)} = (z_{k,j}^{(1)})_{k \neq j} \in \reals^{n (n - 1)}$ are the first layer activations.

    Let $R \in \reals^{n \times n}$ be defined by $\bbra{R}_{ij} = r(x_i, x_j)$. The matrix $R$ specifies the relation function on the sample points $x_1, \ldots, x_n$. We have that $R = P^\top Q,\, P,Q \in \reals^{m \times n}$ for some $m \leq n$. 

    Let $\phi = P \circ \eta$ and $\psi = Q \circ \eta$. We will show that the inner product of neural networks $\iprod{\phi(x)}{\psi(y)}$ approximates $r(x, y)$. In fact, this is immediate. Take $x, y \in \calX$. We have
    \begin{align*}
        \abs{r(x, y) - \iprod{\phi(x)}{\psi(y)}} &\leq \abs{r(x, y) - r(q(x), q(y))} + \abs{r(q(x), q(y)) - \iprod{\phi(x)}{\phi(y)}}\\
        &\leq \epsilon + \abs{r(q(x), q(y)) - \eta(x)^\top R\, \eta(y)},
    \end{align*}
    where the second inequality follows by the assumption of continuity on the relation function $r$ and the choice of the quantization diameter $\delta$. Now, note that whenever $x \in \mathrm{int}(V^{(i)})$, we have $\eta(x) = e_i$, where $e_i$ is the canonical basis vector. Hence $r(q(x), q(y)) = \eta(x)^\top R\, \eta(y)$ Lebesgue-almost everywhere. This completes the proof.
\end{proof}

\begin{remark}
    From a learning perspective, we only need to know the value of the relation function at $n$ (uniformly distributed) points $x_1, \ldots, x_n$, with $n = \calO(\delta^{-d})$ depending on the smoothness of $r$ and the dimension of the space $\calX$.
\end{remark}

\begin{remark}
    In the symmetric case, where the relation function $r$ is assumed to be a positive-definite kernel, Mercer's theorem implies an inner product-like structure in $r$, with a ``rank'' determined by the spectral decay of the kernel. In the asymmetric case, the above result assumes only the continuity of the relation function $r$ and no further ``inner-product-like'' structure. In some applications, the relation function may have a ``low-rank'' structure such that it is representable by an inner product between low-dimensional feature maps, enabling more favorable bounds on the size of the neural network needed, even in the asymmetric case.
\end{remark}

%% BELOW IS AN OLD VERSION OF THIS SECTION WHERE THE CONSTRUCTION IS BASED ON 

% In the previous section we considered symmetric inner product relations where the encoder of the first object is the same as the encoder of the second object. When the underlying relation being modeled is a symmetric `similarity' relation, this is a useful inductive bias. However, in general, relations between objects can be asymmetric. For example, order relations are asymmetric (in fact, anti-symmetric). Such relations cannot be captured by symmetric inner products. In this section, we consider modeling a general (asymmetric) relation $r: \calX \times \calX$ as the inner product of two different neural network encodings of a pair of objects,
% \begin{equation}\label{eq:asymmetric_iprod_relation}
%     r(x, y) = \iprod{\phi(x)}{\psi(y)},
% \end{equation}
% where $\phi, \psi: \calX \to \reals^d$ are two neural networks.

% In this section, we show that inner products of universal approximators (e.g., when $\phi, \psi$ are multi-layer perceptrons) can approximate any continuous function on $\calX \times \calX$.

% We begin with the following simple lemma which states when the object space $\calX$ is finite, any relation function can be represented as the inner product between two encodings.

% \begin{lemma}\label{lemma:finite_space_rel}
%     Suppose $\calX$ is a finite space. Let $r: \calX \times \calX \to \reals$ be any relation function. Then, there exists $d \leq \abs{\calX}$ and $\phi, \psi: \calX \to \reals^{d}$ such that,
%     \begin{equation*}
%         r(x, y) = \iprod{\phi(x)}{\psi(y)}, \ \forall x, y \in \calX.
%     \end{equation*}
% \end{lemma}

% \begin{proof}
%     Let $x_1, \ldots, x_m$ be an enumeration of $\calX$ where $m = \abs{\calX}$. Let $R \in \reals^{m \times m}$ such that $R_{ij} = r(x_i, x_j)$. There exists many decompositions of the matrix $R$ which would induce valid encodings $\phi, \psi$. One example is rank decomposition. Let $d = \mathrm{rank}(R)$. Then, there exists matrices $P, Q \in \reals^{d \times m}$ such that $R = P^\top Q$. Let $\phi, \psi: \calX \to \reals^{d}$ be defined by
%     \begin{equation}
%         \phi(x_i) = P_{i, \cdot}, \ \psi(x_i) = Q_{\cdot, i}, \ \forall i \in [m].
%     \end{equation}

%     Then, $r(x, y) = \iprod{\phi(x)}{\psi(y)}$ for all $x, y \in \calX$.
% \end{proof}

% Note that if each $x \in \calX$ is a one-hot vector in $\reals^{\abs{\calX}}$, then the result above holds with linear maps $\phi, \psi$. Although very simple, this result has direct implications to domains such as language modeling where $\calX$ is a discrete set of tokens, and hence finite. In such cases,~\Cref{lemma:finite_space_rel} tells us that any relation function can be approximated by inner products of feature maps (i.e., of the form present in the attention mechanisms of Transformers). Moreover, in the case of language, there may be a low-rank structure (e.g., depending on syntax, semantics, etc.) enabling a more modest dimension of the feature maps, $d \ll \abs{\calX}$.

% Next, we proceed to show that arbitrary continuous relation functions can be approximated by inner products of two different neural networks. For simplicity, we assume $\calX = [0, 1]^{d}$, where $d = \dim(\calX)$.

% \textbf{Preliminaries.} Let $\calR(\delta) = \sset{R_1, \ldots, R_{\aabs{\calR}}}$ be a partition of $\calX$ into rectangles $R_i = \times_{j=1}^{d} [a_j^i, b_j^i)$ such that $\mathrm{length}(R_i) \leq \delta$ for all $i$. Note that such partition exists with $\delta^{-d}$ rectangles. Let $\norm{\cdot}_\calF$ be a norm on the space of functions from $\calX$ to $\reals$. In particular, we consider the sup-norm $\nnorm{f(\cdot)}_\calF = \sup_{x \in \calX} \abs{f(x)}$ or $L^p$-norms $\nnorm{f(\cdot)}_\calF = \pparen{\int_\calX \abs{f(x)}^p d\mu(x)}^{1/p}$ for some probability measure $\mu$. We denote the corresponding norm on functions from $\calX \times \calX$ to $\reals$ as $\nnorm{f(\cdot)}_{\calF^\otimes 2}$.%, defined as in~\Cref{sec:symmetric_relations}.

% Now consider the quantization function which maps $\calX$ to one of the rectangles in $\calR$. In particular, for a partition of rectangles $\calR(\delta)$, let $q_\delta: \calX \to \reals^{\aabs{\calR}}$ be defined as $q_\delta = \pparen{\bm{1}_{R_1}, \ldots, \bm{1}_{R_{\aabs{\calR}}}}$. That is, for each dimension, it is the indicator of whether the input lies in the corresponding rectangle. We will be interested in approximating $q_\delta$ by a neural network. For a class of neural networks $\calV_N$ with complexity (i.e., total number of units) $N$, let $\mathrm{dist}(q_\delta, \calV_N) = \inf_{h \in \calV_N} \nnorm{q_\delta - h}_\calF$. Let $\calN(\epsilon) \in \naturals$ be the integer such that $\mathrm{dist}(q_\delta, \calV_N) \leq \epsilon$ if $N \geq \calN(\epsilon; \calV)$. We assume that $\calV_N \subset \calV_{N+1}$.

% The following theorem states that any continuous relation function can be approximated by an inner product of two neural networks. The proof strategy will be to quantize the input into rectangles using a neural network, then use~\Cref{lemma:finite_space_rel}.

% \begin{theorem}\label{theorem:asymemtric_inner_prod_rel_func_class}
%     Suppose the relation function $r: \calX \times \calX \to \reals$ is continuous. In particular, for any $\epsilon > 0$, there exists $\delta(\epsilon) > 0$ such that for any $x, y, \tilde{x}, \tilde{y}$ satisfying $\infnorm{x - \tilde{x}} \leq \delta$, $\infnorm{y - \tilde{y}} \leq \delta$, we have $\abs{r(x, y) - r(\tilde{x}, \tilde{y})} \leq \epsilon$. Then, for any approximation error $\epsilon > 0$ there exists multi-layer perceptrons $\phi, \psi$ such that
%     \begin{equation*}
%         \norm{r(x,y) - \iprod{\phi(x)}{\psi(y)}}_{\calF^{\otimes 2}} \leq \epsilon.
%     \end{equation*}
%     Moreover, $\phi, \psi$ can be constructed such that $\phi = L_\phi \circ \hat{q}$, $\psi = L_\psi \circ \hat{q}$ where $\hat{q} \in \calV_N$ is a shared MLP with $N$ neurons and $L_\phi, L_\psi$ are linear projections onto $\delta^{-d}$-dimensional space. The number of neurons is bounded by $N = \calN(\tilde{\epsilon}; \calV)$, where $\tilde{\epsilon} = \pparen{\sqrt{2 \delta^{-d} R_{\max} (2 R_{\max} + \epsilon)} - 2 \delta^{-d / 2} R_{\max}} \pparen{2 \delta^{-d} R_{\max}}^{-1}$,
%     $R_{\max} \coloneq \max_{x,y} r(x,y)$, and $\delta = \delta(\epsilon / 2)$.
% \end{theorem}
% \begin{proof}
%     Let $\delta = \delta(\epsilon / 2)$, and let $\calR(\delta) = {R_1, \ldots, R_{\aabs{\calX}}}$ be a partition of $\calX$ into rectangles, with $\abs{\calR} \leq \delta^{-d}$. Recall that $q_\delta$ is defined as the quantizer of $\calX$ defined in terms of indicators of the rectangles in $\calR$. Let $N = \calN(\tilde{\epsilon}; \calV)$ and let $\hat{q}_\delta \in \calV_N$ be a neural network such that $\norm{\hat{q}_\delta - q_\delta}_\calF \leq \tilde{\epsilon}$, where $\tilde{\epsilon} > 0$ will be determined later. Let $x_1, \ldots, x_{\aabs{\calR}}$ be the midpoints of the rectangles in $\calR$ such that $x_i \in R_i,\, i \in [\aabs{\calR}]$. Let $R \in \reals^{\aabs{\calR} \times \aabs{\calR}}$ be defined by $\bbra{R}_{ij} = r(x_i, x_j)$. We have that $R = P^\top Q,\, P,Q \in \reals^{r \times \aabs{\calR}}$ for some $r \leq \delta^{-d}$.

%     Let $\phi = P \circ \hat{q}_\delta$ and $\psi = Q \circ \hat{q}_{\delta}$. We will show that the inner product of neural networks $\iprod{\phi(x)}{\psi(y)}$ approximates $r(x, y)$. For convenience, we denote $q_\delta(x)$ by $q_\delta(x)$ and $\hat{q}_\delta(x)$ by $\hat{q}_\delta(x)$ in the calculation below.
%     \begin{align*}
%         &\norm{r(x,y) - \iprod{\phi(x)}{\psi(y)}}_{\calF^{\otimes 2}} \\
%         &\leq \norm{r(x, y) - r(q_\delta(x), q_\delta(y))}_{\calF^{\otimes 2}} + \norm{r(q_\delta(x), q_\delta(y)) - \iprod{\phi(x)}{\psi(y)}}_{\calF^{\otimes 2}}\\
%         &\stepa{\leq} \frac{\epsilon}{2} + \norm{q_{\delta}(x)^\top R\, q_{\delta}(y) - \hat{q}_{\delta}(x)^\top R\, \hat{q}_{\delta}(y)}_{\calF^{\otimes 2}}\\
%         &\stepb{\leq} \frac{\epsilon}{2} + \norm{(q_\delta(x) - \hat{q}_\delta(x))^\top R\, q_\delta(y)}_{\calF^{\otimes 2}} + \norm{\hat{q}_\delta(x)^\top R\, (q_\delta(y) - \hat{q}_\delta(y))}_{\calF^{\otimes 2}}
%     \end{align*}
%     In the above, step (a) is by the assumption of continuity of $r: \calX \times \calX \to \reals$ and the definition of the matrix $R$ and the quantizer $q_\delta$. Step (b) is the triangle inequality.

%     We proceed to bound each of the two terms above. The first term is bounded as follows
%     \begin{align*}
%         \abs{(q_\delta(x) - \hat{q}_\delta(x))^\top R\, q_\delta(y)} &\leq \twonorm{q_\delta(x) - \hat{q}_\delta(x)} \twonorm{q_\delta(y)} \norm{R}_{2 \mapsto 2}\\
%         &\leq \twonorm{q_\delta(x) - \hat{q}_\delta(x)} \delta^{-\delta_\calX} R_{\max},
%     \end{align*}
%     where we define $R_{\max} \coloneq \max_{x,y} \aabs{r(x,y)}$ and the second inequality follows by noting that the two-norm of a matrix $A \in\reals^{m \times n}$ is bounded by $\norm{A}_{2 \mapsto 2} \leq \sqrt{m n} \norm{A}_{\max}$. The second term is bounded by first noting that
%     \begin{align*}
%         \abs{\hat{q}_\delta(x)^\top R\, (q_\delta(y) - \hat{q}_\delta(y))} &\leq \twonorm{\hat{q}_\delta(x)^\top R} \twonorm{q_\delta(y) - \hat{q}_\delta(y)} \\
%         &\leq \paren{\twonorm{(\hat{q}_\delta(x) - q_\delta(x))^\top R} + \twonorm{q_\delta(x) R}} \twonorm{q_\delta(y) - \hat{q}_\delta(y)}\\
%         &\leq \paren{\norm{R}_{2 \mapsto 2} \twonorm{\hat{q}_\delta(x) - q_\delta(x)}+ \max_i \twonorm{R_{i\cdot}}} \twonorm{q_\delta(y) - \hat{q}_\delta(y)}\\
%         &\leq \delta^{-d} R_{\max} \twonorm{\hat{q}_\delta(x) - q_\delta(x)} \twonorm{q_\delta(y) - \hat{q}_\delta(y)} + \delta^{-d / 2} R_{\max} \twonorm{q_\delta(y) - \hat{q}_\delta(y)}.
%     \end{align*}
%     Hence,
%     \begin{align*}
%         \norm{\hat{q}_\delta(x)^\top R\, (q_\delta(y) - \hat{q}_\delta(y))}_{\calF^{\otimes 2}} &\leq  \delta^{-d} R_{\max} \norm{\twonorm{\hat{q}_\delta - q_\delta}}_{\calF}^2 + \delta^{-d / 2} R_{\max} \norm{\twonorm{q_\delta - \hat{q}_\delta}}_{\calF}.
%     \end{align*}

%     Putting this together and continuing with the bound, we obtain
%     \begin{align*}
%         &\norm{r(x,y) - \iprod{\phi(x)}{\psi(y)}}_{\calF^{\otimes 2}} \leq \frac{\epsilon}{2} + 2 \delta^{- d / 2} R_{\max} \norm{\twonorm{q_\delta - \hat{q}_\delta}}_{\calF} + \delta^{- d} R_{\max} \norm{\twonorm{q_\delta - \hat{q}_\delta}}_{\calF}^2.
%     \end{align*}

%     Next, we choose the neural network complexity $N$ such that the above is bounded by $\epsilon$. Vt considering the above as a quadratic in $\norm{\twonorm{q_\delta - \hat{q}_\delta}}_{\calF}$, we observe that this holds when
%     \begin{equation*}
%         \norm{\twonorm{q_\delta - \hat{q}_\delta}}_{\calF} \in \Big(0, \frac{\sqrt{2 \delta^{-d} R_{\max} (2 R_{\max} + \epsilon)} - 2 \delta^{-d / 2} R_{\max}}{2 \delta^{-d} R_{\max}} \Big].
%     \end{equation*}
%     This can be seen by considering the quadratic function in the error $\nnorm{\twonorm{q_\delta - \hat{q}_\delta}}_\calF$ and finding values for which the overall error is bounded by $\epsilon$. Hence, letting
%     \[\tilde{\epsilon} \coloneq \frac{\sqrt{2 \delta^{-d} R_{\max} (2 R_{\max} + \epsilon)} - 2 \delta^{-d / 2} R_{\max}}{2 \delta^{-d} R_{\max}}\]
%     and $N = \calN(\tilde{\epsilon}; \calV)$, there exists a neural network $\hat{q}_\delta \in \calV_N$ such that
%     \begin{align*}
%         \norm{r(x,y) - \iprod{\phi(x)}{\psi(y)}}_{\calF^{\otimes 2}} \leq \epsilon
%     \end{align*}
% \end{proof}

% \aanote*{An alternate bound on $\tilde{\epsilon}$... more interpretable but less tight. should we use this instead?}{
% Consider the proof starting from here...
% \begin{equation*}
%     \norm{r(x,y) - \iprod{\phi(x)}{\psi(y)}}_{\calF^{\otimes 2}} \leq \frac{\epsilon}{2} + 2 \delta^{- d / 2} R_{\max} \norm{\twonorm{q_\delta - \hat{q}_\delta}}_{\calF} + \delta^{- d} R_{\max} \norm{\twonorm{q_\delta - \hat{q}_\delta}}_{\calF}^2
% \end{equation*}
% We will try to get a different, more interpretable bound on $\tilde{\epsilon}$. Let $\tilde{\epsilon} := \nnorm{\twonorm{q_\delta - \hat{q}_\delta}}_{\calF}$ and consider the quadratic $2 \delta^{-d/2} R_{\max} \tilde{\epsilon} + \delta^{-d} R_{\max} \tilde{\epsilon}^2 + \epsilon / 2$. We want to find values of $\tilde{\epsilon}$ such that this is bounded by $\epsilon$.

% Assume that $\delta = \min(\delta(\epsilon / 2), 1)$ and $\tilde{\epsilon} \leq 1$. Then,
% \begin{align*}
%     2 \delta^{-d/2} R_{\max} \tilde{\epsilon} + \delta^{-d} R_{\max} \tilde{\epsilon}^2 + \epsilon / 2 &\leq 2 \delta^{-d/2} R_{\max} \tilde{\epsilon} + \delta^{-d} R_{\max} \tilde{\epsilon} + \epsilon / 2\\
%     &\leq 2 \delta^{-d} R_{\max} \tilde{\epsilon} + \delta^{-d} R_{\max} \tilde{\epsilon} + \epsilon / 2\\
%     &= 3 \delta^{-d} R_{\max} \tilde{\epsilon} + \epsilon / 2
% \end{align*}
% where the first inequality is since $\tilde{\epsilon} \leq 1$ and the second is because $\delta \leq 1$. Hence, the expression is less than $\epsilon$ when $\tilde{\epsilon} = \min((6 R_{\max})^{-1} \delta^d \epsilon, 1)$.

% }
\aanote{[todo]: add discussion/section on the scale of NNs needed to approximate a particular relation function.}

\aanote{can we show that $\delta^{-\dim(\calX)}$ is necessary in the worst case? can make use of lower bounds from standard MLPs.}

\aanote{
    this universal approximation is based on arbitrary continuous functions on the $d$-dimensional space. but, in practice, we may be interested in relations with ``low-rank structure'' (i.e., relations in low-dimensional spaces). e.g., $\calX$ may be a space of images (e.g., $d =  l \times w \times 3$), but we may be only interested in computing relations between an attribute like color, texture, or shape of a particular object in the image, which  would be representable in a much lower-dimensional space (e.g., $d = 3$). Moreover, `projections' onto the low-dimensional space on which to compute relations may be very smooth or even linear. e.g., $r(x, y) = \bar{r}(\phi(x), \psi(y))$ where $\phi, \psi$ are smooth and low-dimensional. then, we can get something like $\delta^{-k}$ to approximate $\bar{r}$ and $k \cdot \calN_\calF(\epsilon)$ to approximate $\phi, \psi$ which we can assume are smooth. can we formalize this and develop a theory. i.e., formalize an assumption on the relation function as a composition of a smooth projection into a lower dimensional space followed by a continuous bivariate function. can note relation to how `relations' are computed in Transformers/Abstractors/RelConvNet, etc. first a FeedForward net, then low-dimensional projections.
    }
\section{Connection to reproducing kernel Hilbert and Banach spaces}\label{sec:rkbs_asymmetric_relations}

In~\Cref{sec:symmetric_relations} we showed that the function class of symmetric inner products of neural networks is the set of symmetric positive-definite kernels---that is, reproducing kernels of reproducing kernel Hilbert spaces (RKHS). There exists a similar interpretation of the function class of asymmetric inner products of neural networks in terms of the reproducing kernels of \textit{reproducing kernel Banach spaces} (RKBS).

Recall that a reproducing kernel Hilbert space $\calH$ is a Hilbert space of functions on a space $\calX$ for which the point evaluation functionals $f \mapsto f(x)$ are continuous. \citet{aronszajn1950theory} showed that there is a one-to-one identification between RKHSs and symmetric positive definite kernels $K: \calX \times \calX \to \reals$ such that $\iprod{K(x, \cdot)}{f}_\calH = f(x)$.~\citet{mercerFunctionsPositive1909} had previously shown that an RKHS can be identified with a feature map via the spectral decomposition of the integral operator $T_K: L_2(\calX) \to L_2(\calX)$ defined by $T_K f(x) = \int_\calX K(x, y) f(y) dy$. Every feature map $\phi: \calX \to \calW$ defines a symmetric positive definite kernel $K(x, y) = \iprod{\phi(x)}{\phi(y)}_\calW$ (and hence, an RKHS), while every symmetric positive definite kernel has infinitely many feature map representations.

\Cref{theorem:symmetric_inner_prod_rels_func_class} shows that the function class of symmetric inner products of neural networks is the set of reproducing kernels of RKHS function spaces. A reproducing kernel Hilbert space is, as the name suggests, a \textit{Hilbert space} of functions on some space $\calX$. The linear structure of a Hilbert space makes the kinds of geometries it can capture relatively restrictive, since any two Hilbert spaces with the same dimension are isometrically isomorphic. Banach spaces, which have fewer structural assumptions, can capture richer geometric structures. Hence, a reproducing kernel Banach space can capture richer geometries between functions than an RKHS. In particular, in contrast to an RKHS, the reproducing kernel of an RKBS need not be symmetric or positive definite. In this section, we show that the function class of asymmetric inner products of neural networks has an interpretation in terms of the reproducing kernels of RKBSs, mirroring the result for symmetric inner products of neural networks. Nonsymmetric kernels of positive type, a more restricted class, were studied by \citet{seelyNonSymmetricKernels1919}, shortly after Mercer's seminal work for the symmetric case. % \citep{mercerFunctionsPositive1909}.
%\footnote{While working on extending~\Cref{theorem:symmetric_inner_prod_rels_func_class} to the asymmetric case, we came across an interesting paper from 1919 which studied non-symmetric kernels of positive type~\parencite{seelyNonSymmetricKernels1919}. Mercer's work from a decade earlier studied \textit{symmetric} kernels of positive type---the proof of~\Cref{theorem:symmetric_inner_prod_rels_func_class} relied on the property of uniform convergence of the series of characteristic functions: $\sum_i \lambda_i \psi_i(s)\psi_i(t) \to K(s,t)$. Ultimately, we didn't use the results of~\parencite{seelyNonSymmetricKernels1919} since asymmetric neural network inner products need not be of positive-type (the function class is in fact larger).}.

\subsection{Background on reproducing kernel Banach spaces}

\begin{definition}[Reproducing Kernel Banach Space]
    A \textbf{reproducing kernel Banach space} on a space $\calX$ is a Banach space $\calB$ of functions on $\calX$, satisfying:
    \begin{enumerate}
        \item $\calB$ is \textit{reflexive}. That is, $(\calB^*)^* = \calB$, where $\calB^*$ is the dual space of $\calB$. Furthermore, $\calB^*$ is isometric to a Banach space $\calB^\#$ of functions on $\calX$.
        \item The point evaluation functionals $f \mapsto f(x)$ are continuous on both $\calB$ and $\calB^\#$.
    \end{enumerate}
\end{definition}

This definition is a strict generalization of reproducing kernel Hilbert spaces, as any RKHS $\calH$ on $\calX$ is also an RKBS, since (1) is implied by the Riesz representation theorem. While the identification $\calB^\#$ is not unique, we can choose some identification arbitrarily and denote it by $\calB^*$ for ease of notation (by assumption, all identifications are isometric to each other). Thus, if $\calB$ is an RKBS, $\calB^*$ is also an RKBS.

Similar to an RKHS, an RKBS also has a \textit{reproducing kernel}. To state the result, for a normed vector space $\calV$ and its dual space $\calV^*$, we define the bilinear form
\begin{equation}\label{eq:bilinear_form}
    \begin{split}
        \calV \times \calV^* &\to \reals\\
        (u, v^*)_\calV &\mapsto v^*(u).
    \end{split}
\end{equation}

Theorem 2 of \citet{zhangReproducingKernel2009} shows that for any RKBS $\calB$ there exists a unique reproducing kernel $K: \calX \times \calX \to \bbC$ that recovers point evaluations, meaning
\begin{align}
    f(x) &= \paren{f, K(\cdot, x)}_\calB, \quad \forall f \in \calB, \\
    f^*(x) &= \paren{K(x, \cdot), f^*}_\calB,\quad \forall f^* \in \calB^*,
\end{align}

and such that the span of $K(x, \cdot)$ is dense in $\calB$ and the span of $K(\cdot, x)$ is dense in $\calB^*$,
\begin{align}
    \overline{\text{span}}\{K(x, \cdot): x \in \calX\} &= \calB, \\
    \overline{\text{span}}\{K(\cdot, x): x \in \calX\} &= \calB^*.
\end{align}
Finally,
\begin{equation}
    K(x, y) = \paren{K(x, \cdot), K(\cdot, y)}_\calB, \ \forall x, y \in \calX.
\end{equation}
Unlike RKHSs, while each RKBS has a unique reproducing kernel, different RKBSs may have the same reproducing kernels.

Furthermore, a kernel $K: \calX \times \calX \to \bbC$ is the reproducing kernel of some RKBS if and only if it has a feature map representation. Crucially for us, the feature map representation is more versatile than the one for RKHSs. Let $\calW$ be a reflexive Banach space with dual space $\calW^*$. Consider a pair of feature maps $\Phi$ and $\Phi^*$, mapping to each feature space, respectively. That is,
\begin{equation*}
    \Phi: \calX \to \calW, \ \Phi^*: \calX \to \calW^*,
\end{equation*}
where we call $\Phi,\, \Phi^*$ the \textit{pair} of feature maps and $\calW,\, \calW^*$ the pair of feature spaces. Suppose that the span of the image of the feature maps under $\calX$ is dense in their respective feature spaces. That is,
\begin{equation}
    \overline{\text{span}}\{\Phi(x): x \in \calX\} = \calW, \;\; \overline{\text{span}}\{\Phi^*(x): x \in \calX\} = \calW^*.
\end{equation}
Then, by Theorem 3 of \citet{zhangReproducingKernel2009}, the feature maps $\Phi, \Phi^*$ induce an RKBS defined by
\begin{align}
    \calB &:= \set{f_w: x \mapsto (\Phi^*(x))(w), w \in \calW} \\
    \norm{f_w}_\calB &:= \norm{w}_\calW,
\end{align}
with the dual space $\calB^*$ defined by
\begin{align}
    \calB^* &:= \set{f_{w^*}: x \mapsto w^*(\Phi(x)), w^* \in \calW^*} \\
    \norm{f_{w^*}}_{\calB^*} &:= \norm{w^*}_{\calW^*}.
\end{align}
Furthermore, for any RKBS, there exist some feature spaces $\calW, \calW^*$ and feature maps $\Phi, \Phi^*$ such that the above construction yields that RKBS, which is Theorem 4 of \citet{zhangReproducingKernel2009}.

\subsection{Asymmetric inner products of neural networks model kernels of reproducing kernel Banach spaces}

Observe that for an RKBS with feature-map representation given by $\Phi, \Phi^*$, its reproducing kernel is given by
\begin{equation}
    K(x, y) = \paren{\Phi(x), \Phi^*(y)}_{\calW}, \ x, y \in \calX,
\end{equation}
where $\paren{\cdot, \cdot}_{\calW}$ is the bilinear form on $\calW$ defined in~\Cref{eq:bilinear_form}.

This form is reminiscent of the asymmetric inner product of neural networks,
\begin{equation}
    r(x, y) = \iprod{\phi(x)}{\psi(y)}, \ x, y \in \calX,
\end{equation}
where $\phi, \psi: \calX \to \reals^{d}$ is a pair of learned feature maps. We will show that asymmetric inner products of neural networks can approximate any reproducing kernel of an RKBS.

Let $\Phi, \Phi^*$ be a pair of feature maps that define the reproducing kernel $r(x,y) = \pparen{\Phi(x), \Phi^*(y)}_\calW$ for some RKBS. Recall that any two Hilbert spaces with equal dimensions are isometrically isomorphic. Hence, when the feature space $\calW$ is a Hilbert space, we can consider $\calW = \ell^2(\naturals)$ without loss of generality. %Note that $\ell^2(\naturals)$ is self-dual. Hence, by the Riesz representation theorem, there exists a unique element $\sigma(\Phi^*(y)) \in \ell^2(\bbN)$ such that $r(x,y) = \iprod{\Phi(x)}{\sigma \circ \Phi^*(y)}_{\ell^2}$.

The following theorem states that when $\calB$ is an RKBS on $\calX$ with a feature map representation whose feature space $\calW$ is a Hilbert space, its reproducing kernel can be approximated by an asymmetric inner product of neural networks. The proof is similar to that of~\Cref{theorem:symmetric_inner_prod_rels_func_class}.

\begin{theorem}\label{thm:asymmetric_inner_prod_approximates_rkbs}
   Suppose $\calX$ is a compact metric space. Suppose $r: \calX \times \calX \to \reals$ is the reproducing kernel of some RKBS $\calB$ on $\calX$ admitting a continuous feature map representation with a feature space $\calW$ that is a Hilbert space. 
   Then, for any $\varepsilon > 0$, there exist multi-layer perceptrons 
    $\phi, \psi: \calX \to \reals^{d}$ such that 
   \begin{equation*}
        \sup_{x,y \in \calX} \abs{r(x, y) - \iprod{\phi(x)}{\psi(y)}} \leq \varepsilon.
   \end{equation*}
\end{theorem}

\begin{proof}
    By assumption, there exists a Hilbert space $\calW$ and a pair of feature maps $\Phi: \calX \to \calW, \Phi^*: \calX \to \calW$ such that,
    \begin{equation*}
        r(x, y) = \paren{\Phi(x), \Phi^*(y)}_{\calW} \equiv (\Phi^*(y))(x), \ x, y \in \calX.
    \end{equation*}

    Without loss of generality, we can restrict our attention to the feature space $\calW = \ell^2(\bbN)$, since any two Hilbert spaces with equal dimension are isometrically isomorphic. The dual space is $\calW^* = \ell^2(\bbN)$. Hence, for feature maps $\Phi, \Phi^*$, the ground truth relation to be approximated is,
    \begin{equation*}
        r(x, y) = \paren{\Phi(x), \Phi^*(y)}_{\ell^2(\bbN)} \equiv (\Phi^*(y))(\Phi(x)), \ x, y \in \calX.
    \end{equation*}
    By the Riesz representation theorem, there exists a unique element in $u_{\Phi^*(y)} \in \ell^2(\bbN)$ such that,
    \begin{equation*}
        (\Phi^*(y))(w) = \iprod{w}{u_{\Phi^*(y)}}_{\ell^2(\bbN)}, \ \forall w \in \ell^2(\bbN).
    \end{equation*}
    Let $\sigma: \ell^2(\bbN)^* \to \ell^2(\bbN)$ denote the mapping from an element in the dual space to its Riesz representation. Then $\sigma$ is a bijective isometric antilinear isomorphism; the Riesz representation can be constructed via an orthonormal basis through $\sigma(w^*) = \sum_{i \in I} w^{*}(e_i) e_i$, where $\set{e_i}_{i \in I}$ is some basis for $\calW$.

    Thus, the relation function on $\calX \times \calX$ that we need to approximate is
    \begin{equation*}
        r(x, y) = \iprod{\sigma \circ \Phi^* (y)}{\Phi(x)}_{\calW}, \ x, y \in \calX.
    \end{equation*}
    We do this by approximating $\Phi: \calX \to \calW$ with the MLP $\psi$ and approximating $\sigma \circ \Phi^*: \calX \to \calW$ with the MLP $\phi$.

    First, since $\Phi(x), \sigma \circ \Phi^*(y) \in \ell^2(\bbN), \forall x, y$, and $\calX$ is compact, we have
    \begin{equation*}
        \lim_{n \to \infty} \sup_{x,y \in \calX} \abs{r(x, y) - \sum_{i=1}^{n} (\Phi(x))_i \cdot (\sigma(\Phi^*(y)))_i} = 0.
    \end{equation*}
    Thus, let $d$ be such that,
    \begin{equation}\label{eq:thm1_proof_eq1}
        \sup_{x,y \in \calX} \abs{r(x, y) - \sum_{i=1}^{d} (\Phi(x))_i \cdot (\sigma(\Phi^*(y)))_i} < \frac{\varepsilon}{2}.
    \end{equation}
    We will obtain MLPs $\phi,\, \psi$ that are functions from $\calX$ to $\reals^{d}$. 
    %Let $\paren{(\Phi(x))_1, \ldots, (\Phi(x))_{d}}$ be the function to be approximated by the MLP $\phi_\theta$ and let $\paren{(\sigma(\Phi^*(y)))_1, \ldots, (\sigma(\Phi^*(y)))_{d}}$ be the function to be approximated by the MLP $\psi$. 
    By the universal approximation property of MLPs, for any $\tilde{\varepsilon} > 0$, there exists MLPs $\phi,\,\psi$ such that
    \begin{equation}\label{eq:thm1_proof_eq2}
        \sup_{x \in \calX} \abs{(\phi(x))_i - (\Phi(x))_{i}} < \tilde{\varepsilon} \ \text{ and } \ \sup_{x \in \calX} \abs{(\psi(y))_i - (\sigma(\Phi^*(x)))_{i}} < \tilde{\varepsilon}, \ \forall i \in \set{1, \ldots, d}.
    \end{equation}
    For example, \citet{cybenkoApproximationSuperpositions1989} shows that 1-layer neural networks with discriminatory activation functions of the form $\sum_{i=1}^{N} \alpha_i \sigma(w_j^\top x + b_j)$ are dense in the space of continuous functions.
    Now,
    \begin{align*}
        &\sup_{x,y \in \calX} \abs{r(x,y) - \hat{r}(x,y)} \\
        &= \sup_{x,y \in \calX} \abs{r(x,y) - \iprod{\phi(x)}{\psi(y)}} \\
        &\leq \sup_{x,y\in \calX} \paren{\abs{r(x,y) - \sum_{i=1}^{d} (\Phi(x))_i \cdot (\sigma(\Phi^*(y)))_i} + \abs{\sum_{i=1}^{d} (\Phi(x))_i \cdot (\sigma(\Phi^*(y)))_i - \iprod{\phi(x)}{\psi(y)}}}.
    \end{align*}
    The first term is less than $\varepsilon / 2$ by~\Cref{eq:thm1_proof_eq1}. Now, we bound the second term uniformly on $x,y \in \calX$, as
    \begin{align*}
        &\abs{\sum_{i=1}^{d} (\Phi(x))_i \cdot (\sigma(\Phi^*(y)))_i - \iprod{\phi(x)}{\psi(y)}} \\
        &\leq \sum_{i=1}^{d} \abs{(\Phi(x))_i \cdot (\sigma(\Phi^*(y)))_i - (\phi(x))_i(\psi_i(y))_i} \\
        &\leq \sum_{i=1}^{d} \paren{\abs{(\sigma(\Phi^*(y)))_i} \abs{(\sigma(\Phi^*(y)))_i - (\psi_i(y))_i} + \abs{(\Phi(x))_i} \abs{(\Phi(x))_i - (\phi(x))_i}}\\
        &\leq C(\Phi, \Phi^*, d) \sum_{i=1}^{d} \paren{\abs{(\sigma(\Phi^*(y)))_i - (\psi_i(y))_i} + \abs{(\Phi(x))_i - (\phi(x))_i}},
    \end{align*}
    where $C(\Phi, \Phi^*, d) \coloneq \max\sset{\max_{y \in \calX, i \in [d]} \aabs{(\sigma(\Phi^*(y)))_i}, \max_{x \in \calX, i \in [d]} \aabs{\abs{(\Phi(x))_i}}}$. Let the approximation error $\tilde{\varepsilon}$ in~\Cref{eq:thm1_proof_eq2} be small enough such that the above is smaller than $\varepsilon / 2$. This shows that
    \begin{equation*}
        \sup_{x,y \in \calX} \abs{r(x,y) - \tilde{r}_i(x,y)} \leq \frac{\varepsilon}{2} + \frac{\varepsilon}{2} = \varepsilon.
    \end{equation*}
\end{proof}

\begin{remark}
    The reason we assume that the underlying RKBS $\calB$ admits a feature map representation with feature space $\calW$ that is a Hilbert space is so that we can use the Riesz representation theorem. The Riesz representation theorem is what links the broad framework of reproducing kernel Banach spaces back to the inductive bias of modeling relations as inner products of feature maps.
\end{remark}

\begin{remark}
    \citet{zhangReproducingKernel2009} explore a specialization of reproducing kernel Banach spaces in which $\calB$ has a semi-inner product. This added structure grants semi-inner product RKBSs some desirable properties that RKHSs have but general RKBSs lack (e.g., convergence in the space implies pointwise convergence, weak universality of kernels, etc.). However, their notion of a semi-inner product is too restrictive to allow for our model $\iprod{\phi(x)}{\psi(x)}$.
\end{remark}

\section{Discussion}

The analysis in this note underscores the importance of kernels for learning relations and attention mechanisms. In the symmetric case, the assumption of a positive-definite kernel function is natural, leading to the standard framework of reproducing kernel Hilbert spaces. In the asymmetric case, which is arguably more important and applicable for relational learning, a different technical approach is needed, and reproducing kernel Banach spaces arise naturally. After completing the work presented here, we became aware of the related work of \citet{wright2021transformers}, which makes this connection as well.

The results presented here can be extended in several ways. For example, the bounds on the 
number of neurons in a perceptron that suffice to approximate a relation function to a given accuracy can likely be sharpened, drawing on the extensive literature on approximation properties of neural networks \citep[e.g.,][]{petrushev1998approximation,pinkus1999approximation,makovoz1998uniform,burger2001error,maiorov2006approximation,bachBreakingCurseDimensionality2016}. In terms of attention mechanisms in transformers, our initial focus was on approximating the most relevant key to a given query. The representation theorem of \citet{debreuRepresentationPreferenceOrdering1954} is used to express the problem in terms of a utility function, which is then approximated. It would be of interest to derive approximation bounds for the full distribution of attention values that are computed by the softmax function in Transformers. Finally, when considering relational learning, the possibility of higher-order, recursive relations, naturally arises \citep[e.g.,][]{altabaaRelationalConvolutionalNetworks2023}, and it may be interesting to study function spaces of hierarchical relations in such settings.


\setlength\bibitemsep{7pt}
\printbibliography

\end{document}