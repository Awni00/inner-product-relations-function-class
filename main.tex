\documentclass[12pt,pdftex,noinfoline]{imsart}

\RequirePackage[OT1]{fontenc}
\usepackage{mymathstyle}


\makeatletter
\let\c@author\relax
\makeatother
\usepackage[
    style=authoryear-comp,
    doi=false,
    url=false,
    isbn=false,
    eprint=false,
    backend=bibtex]{biblatex}
\addbibresource{references.bib}
\renewbibmacro*{volume+number+eid}{%
  \printfield{volume}%
%  \setunit*{\adddot}% 
  \setunit*{\addnbspace}% 
  \printfield{number}%
  \setunit{\addcomma\space}%
  \printfield{eid}}
\DeclareFieldFormat[article]{number}{\mkbibparens{#1}}
\usepackage{xpatch}
\xpatchbibdriver{article}
  {\usebibmacro{title}%
   \newunit}
  {\usebibmacro{title}%
   \printunit{\addcomma\space}}
  {}
  {}
\xpatchbibdriver{inproceedings}
  {\usebibmacro{title}%
   \newunit}
  {\usebibmacro{title}%
   \printunit{\addcomma\space}}
  {}
  {}
\setlength\bibitemsep{7pt}
\renewbibmacro{in:}{}

\def\given{\,|\,}
\def\P{\mathbb{P}}
\def\E{\mathbb{E}}
\def\reals{\mathbb{R}}
\let\what\widehat
\let\hat\widehat
\let\tilde\widetilde
\let\phi\varphi
\let\epsilon\varepsilon

\renewcommand{\argmin}{\mathop{\rm argmin}}
\newcommand{\ess}{\mathop{\rm ess}}
\renewcommand{\argmax}{\mathop{\rm argmax}}

\begin{document}
\def\snote#1{${}^{#1}$}
\setlength{\parskip}{0.5em}
\begin{frontmatter}
{\bf\Large A note on universal approximation of \\inner products of neural networks for modeling relations}
%\affil[**]{Department of Statistics and Data Science, Yale University}
\begin{aug}
\vskip15pt
\address{
\begin{tabular}{ccccc}
{\normalsize\rm\bfseries Awni Altabaa}\snote{1} & {\normalsize\rm\bfseries John Lafferty}\snote{2}\\[5pt]
\end{tabular}
\vskip5pt
\footnotetext{
\snote{1}Department of Statistics and Data Science, Yale University; awni.altabaa@yale.edu.
\snote{2}Department of Statistics and Data Science, Wu Tsai Institute, Institute for Foundations of Data Science, Yale University; john.lafferty@yale.edu.
}
\today
\vskip10pt
}
\begin{abstract}
    Inner products of neural network feature maps arises in a wide variety of machine learning frameworks as a method of modeling relations between inputs.
    This work studies the approximation properties of inner products of neural networks.
    It is shown that the inner product of a multi-layer perceptron with itself is a universal approximator for symmetric positive-definite relation functions. In the case of asymmetric relation functions, it is shown that the inner product of two different multi-layer perceptrons is a universal approximator.
    In both cases, a bound is obtained on the number of neurons required to achieve a given accuracy of approximation. In the symmetric case, the function class can be identified with kernels of reproducing kernel Hilbert spaces, whereas in the asymmetric case the function class can be identified with kernels of reproducing kernel Banach spaces.
    Finally, these approximation results are applied to analyzing the attention mechanism underlying Transformers, showing that any retrieval mechanism defined by an abstract preorder can be approximated by attention through its inner product relations.
    This result uses the Debreu representation theorem in economics to represent preference relations in terms of utility functions.
\end{abstract}
\end{aug}
\end{frontmatter}

\section{Introduction}\label{sec:intro}

\aanote[margin, noinline]{update abstract?}

Machine learning systems must be able to represent and reason about relations between objects, either explicitly or implicitly. For example, a natural language understanding system takes a sequence of words as input and extracts information about the meaning of the words based on relations between them in the local context. Similarly, a scene analysis system considers the relations between the components of a scene in order to identify and interpret the objects.

A common way to represent relations between objects is through inner products between feature representations of the form $\iprod{\phi(x)}{\psi(y)}$, where $x, y \in \calX$ are two objects and $\phi, \psi$ are neural network feature maps. Inner products posses properties which make them useful measures of similarity. The aim of this paper is to understand the representational power of this model by characterizing the class of relation functions $r: \calX \times \calX \to \reals$ which can be represented as inner products of neural networks.

The use of inner products between feature maps is widespread in machine learning architectures. A notable example is the attention mechanisms that lie at the heart of sequence models like the Transformer~\parencite{vaswani2017attention}. In the Transformer, self-attention is implemented as
\begin{equation*}
    \begin{split}
        \alpha_{ij} &\gets \mathrm{Softmax}\paren{\bra{{\iprod{\phi_q(x_i)}{\phi_k(x_j)}}}_{j \in [n]}}_j\\
        x_i' &\gets \sum_{j=1}^{n} \alpha_{ij} \phi_v(x_j)
    \end{split}
\end{equation*}
where $\phi_q, \phi_k$, and $\phi_v$ are learned transformations and $\iprod{\phi_q(x_i)}{\phi_k(x_j)}$ represents a relation between $x_i$ and $x_j$, which determines how much $i$ should attend to $j$. A similar mechanism is at play in earlier neural architectures which implement a content-addressable external memory~\parencite{gravesNeuralTuringMachines2014,gravesHybridComputingUsing2016a,pritzelNeuralEpisodicControl2017}, wherein the read/write operations are typically implememnted using an inner product-based similarity computation followed by a softmax-normalization. Since the Transformer, attention has also been used in other architectures to model relations between different entities~\parencite{velickovicGraphAttentionNetworks2017a,santoroRelationalRecurrentNeural2018,zambaldiDeepReinforcementLearning2018a,locatelloObjectCentricLearningSlot2020b}. For example, \citet{santoroRelationalRecurrentNeural2018} propose a recurrent neural network with a memory module which employs dot product attention to allow memories to interact and model relations within the memory.

The Transformer models relations implicitly through its attention mechanism. Modeling relations through inner products of features is also central to many ``explicitly relational'' neural architectures~\parencite[e.g.,][]{webbEmergentSymbols2021,kergNeuralArchitecture2022,altabaaAbstractorsTransformer2023,altabaaRelationalConvolutionalNetworks2023}. For example, in the model proposed by~\cite{kergNeuralArchitecture2022}, a similarity matrix is computed consisting of symmetric inner products between each pair of objects, $R_{i,\cdot} = \mathrm{Softmax}\pparen{\bra{\iprod{\phi(x_i)}{\phi(x_j)}}_{j\in[n]}}$.~\cite{altabaaAbstractorsTransformer2023} propose a Transformer-based architecture imbued with relational inductive biases by replacing the values $\phi_v(x_i)$ with vector representations which identify objects but are independent of object-level features.~\cite{altabaaRelationalConvolutionalNetworks2023} propose a relational architecture where the central operation is a type of ``convolution'' operating on a tensor of relations computed via inner products of feature maps.

Siamese networks~\parencite{rumelhartLearningRepresentationsBackpropagating1986,langTimedelayNeuralNetwork1988,bromleySignatureVerificationUsing1993,baldiNeuralNetworksFingerprint1993,chopraLearningSimilarityMetric2005,kochSiameseNeuralNetworks2015} are another domain where understanding the function classes of inner products of neural network transformations is relevant. Siamese networks consist of two identical copies of a neural network, with shared parameters, which process two inputs independently producing feature vectors which are then compared using some distance metric to determine the similarity or dissimilarity between the inputs. If the distance is the Euclidean distance, then $\twonorm{\phi(x) - \phi(y)}^2 = \iprod{\phi(x)}{\phi(x)} + \iprod{\phi(y)}{\phi(y)} - 2 \iprod{\phi(x)}{\phi(y)}$, where again the inner product of neural networks arises.

\aanote[margin, noinline]{add brief overview of universal approximation results for MLPs? Mention classical references such as Barron, Cybenko, etc., and more recent analysis such as Bach, etc.?}
% universal approximation of feedforward neural networks has been extensively studied (e.g., classical references are ... Barron, Cybenko, etc.)


In this paper we characterize the function class of inner products of neural networks, showing that inner products of neural networks are universal approximators for relation functions. In particular, when the inner product of neural networks is symmetric (i.e., $\phi=\psi$), the function class is the class of continuous positive definite kernels. When the inner product is asymmetric (i.e., $\phi \neq \psi$), the function class is all continuous bivariate functions on $\calX \times \calX$. The symmetric case is related to kernels of reproducing kernel Hilbert spaces whereas the asymmetric case admits a connection to kernels of reproducing kernel Banach spaces. We apply these results to analyzing the attention operation, showing that any retrieval function described by arbitrary pre-order define 
\section{Function class of symmetric inner product relations}\label{sec:symmetric_relations}

Consider a symmetric relation $r: \calX \times \calX \to \reals$ modeled as the inner product of neural network encodings of a pair of inputs, $x, y \in \calX$,
\begin{equation}\label{eq:symmetric_iprod_relation}
    r(x, y) = \iprod{\phi(x)}{\phi(y)},
\end{equation}
where $\phi$ is a neural network. Symmetric relation functions modeled in this way are natural \textit{measures of similarity}. Intuitively, $\phi$ extracts features of the objects $x,y$, and the inner product computes the similarity between those features. To see this more formally, suppose we have a normalized inner product relation (i.e., $\phi: \calX \to \bbS^{d}$), then we have
\begin{equation}
    \twonorm{\phi(x) - \phi(y)}^2 = 2 - 2 r(x, y).
\end{equation}

% The Cauchy-Schwartz inequality states that
% \begin{equation}
%     r(x, y)^2 \leq r(x, x) r(y, y) = 1, \ \forall x, y \in \calX.
% \end{equation}

Thus, $r(x,y)$ is large and close to $1$ when $x$ and $y$ are similar (with respect to $\phi$), and close to $0$ when $x, y$ are dissimilar. Moreover, $r: \calX \times \calX \to \reals$ induces a geometry on $\calX$ via the pseudometric $d(x,y) = \sqrt{2 - 2 r(x,y)}$. The triangle inequality of the pseudometric corresponds to a notion of transitivity---if $x$ is related to $y$ and $y$ is related to $z$, then $x$ must be related to $z$.

In this section, we characterize the class of relation functions that can be modeled by symmetric inner products of neural networks. We show that any continuous symmetric positive definite kernel (i.e., Mercer kernel) can be approximated by a symmetric inner product of neural networks. This characterization is strict since $r(x,y) = \iprod{\phi(x)}{\phi(y)}$ is a Mercer kernel for any $\phi$ because $\forall g \in L^2(\calX)$ we have,
\begin{equation*}
    \begin{split}
        \int \int g(x) r(x,y) g(y) dx dy &= \int \int g(x) \iprod{\phi(x)}{\phi(y)} g(y) dx dy \\
        &= \iprod{\int g(x) \phi(x) dx}{\int g(y) \phi(y) dy} \\
        &= \twonorm{\int g(x) \phi(x) dx}^2 \geq 0.
    \end{split}
\end{equation*}

The main result in this section states that if $\phi$ is a universal approximator such as a multi-layer perceptron, then for any positive-definite symmetric kernel $r$, there exists a neural network $\phi$ such that the induced inner product relation approximates $r$ arbitrarily well. To state the result, we begin with some preliminaries.

\textbf{Preliminaries: symmetric positive definite kernels.} Consider a symmetric positive definite kernel $K: \calX \times \calX \to \reals$ on a compact euclidean space $\calX$. By Mercer's theorem~\parencite{mercerFunctionsPositive1909, sunMercerTheorem2005, micchelliUniversalKernels2006}, there exist non-negative eigenvalues $\set{\lambda_i}_{i=1}^{\infty}$ and continuous eigenfunctions $\set{\psi_i}_{i=1}^{\infty}$ such that for any $x,y \in \calX$, $K(x, y) = \sum_{i=1}^\infty \lambda_i \psi_i(x) \psi_i(y)$, and the convergence of the series is absolute and uniform over $\calX$. The following definition characterizes, in a sense, the complexity of a symmetric positive-definite kernel in terms of its spectrum.

\begin{assumption}[Kernel spectrum decay]\label{ass:sym_pd_ker_specturm_decay}
	Let $K: \calX \times \calX \to \reals$ be a symmetric positive-definite kernel, with eigenvalues $\{\lambda_i\}_{i=1}^{\infty}$ and continuous eigenfunctions $\{\psi_i\}_{i=1}^{\infty}$. For $\epsilon > 0$, denote by $d_K(\epsilon) \in \naturals$ the minimum integer such that,
	\begin{equation*}
		\sup_{x, y \in \calX} \abs{K(x, y) - \sum_{i=1}^{d_K(\epsilon)} \lambda_i \psi_i(x) \psi_i(y)} \leq \epsilon.
	\end{equation*}
	Moreover, let $C(K, d) \coloneq \max_{i \in [d]} \max_{x \in \calX} \abs{\sqrt{\lambda_i} \psi_i(x)}$.% and let $L(K, d) \coloneq \max_{i \in [d]} L_i$, where $L_i$ is the Lipschitz constant of $\sqrt{\lambda_i} \psi_i$.
\end{assumption}

The quantity $d_K(\epsilon)$ associated with a kernel $K$ describes the number of eigenfunctions needed to approximate $K$ up to an error of $\epsilon$. For any symmetric positive-definite kernel $K$, $d_K(\epsilon)$ is finite for any $\epsilon > 0$. 
% Similarly, the quantity $L(K, d)$ describes the level of ``regularity'' of the first $d$ eigenfunctions of $r$. 
Note that $C(K, d)$ is finite for any $d$ since $\calX$ is assumed compact and the eigenfunctions $\psi_i$ are continuous. % Moreover, $d_K(\epsilon)$ increases as $\epsilon \to 0^+$, and $C(K, d), L(K, d)$ are increasing in $d$. Hence, $C(K, d_K(\epsilon)), L(K, d_K(\epsilon))$ is decreasing in $\epsilon$.

\textbf{Preliminaries: universal approximation of neural networks.} Let $\calV_N$ be a class of neural networks with ``complexity'' $N$. In this work, we take $N$ to be the total number of neurons. We assume that $\calV_{N+1} \supset \calV_N$, and denote $\calV = \cup_N{\calV_N}$. Let $\calF$ be a space of functions from $\calX$ to $\reals$ and let $\nnorm{\cdot}_\calF$ be a norm on functions from $\calX$ to $\reals$. For example, $\nnorm{\cdot}_\calF$ may be the sup-norm or an $L^p$ norm, and $\calF$ may be a class of functions of a particular smoothness. For multivariate functions from $\calX^n$ to $\reals$, we denote the corresponding norm by $\nnorm{\cdot}_{\calF^{\otimes n}}$. In the case of the sup-norm, it is $\sup_{\bm{x} \in \calX^n} \abs{f(x_1, \ldots, x_n)}$ and in the case of a $L^p$ norm it is $\pparen{\int_{\calX^n} \abs{f(x_1, \ldots, x_n)}^p d\mu^{\otimes n}(x_1, \ldots, x_n)}^{1/p}$. We will consider approximating the relation function $r: \calX \times \calX \to \reals$ with respect to $\nnorm{\cdot}_{\calF^{\otimes 2}}$.

Universal approximation results provide theoretical guarantees about how well a class of networks of a given complexity can approximate a given function with respect to $\nnorm{\cdot}_\calF$. In particular, for a given function $f$, universal approximation results characterize the quantity
\[\mathrm{dist}(f, \calV_N) = \inf_{g \in \calV_N} \norm{f - g}_{\calF}.\]
Alternatively, this can be stated as follows: for a desired approximation error $\epsilon$, how large does $N$ need to be so that there exists $g \in \calV_N$ such that $\norm{f - g}_{\calF} \leq \epsilon$? We state this below as an assumption on the network class $\calV$.

\begin{assumption}[Efficiency of function approximator]\label{ass:univ_approx_efficiency}
	Consider a class of neural networks $\calV = \cup_N \calV_N$ on a space $\calX$ where $N$ is a scalable complexity. Let $\calF$ be a class of functions on $\calX$. For $\epsilon > 0$, denote by $\calN_\calF(\epsilon) \in \naturals$ the minimum integer such that for any function $f \in \calF$, there exists a neural network $g \in \calV_N$ with $N = \calN_\calF(\epsilon)$ that approximates $f$ with respect to $\norm{\cdot}_\calF$ with error bounded by $\epsilon$. That is,
	\begin{equation*}
		\sup_{f \in \calF} \, \inf_{g \in \calV_N} \norm{f - g}_{\calF} \leq \epsilon.
	\end{equation*}
	We assume a universal approximation property on the class of neural networks $\calV$: for every $\epsilon > 0$, $\calN_\calF(\epsilon)$ is finite.
\end{assumption}

We are now ready to formally state the main result of this section.

\begin{theorem}[Function class of symmetric inner product relational neural networks]\label{theorem:symmetric_inner_prod_rels_func_class}
	Suppose the data lies in a compact Euclidean space $\mathcal{X}$. Let $\calF$ be a function space from $\calX$ to $\reals$, and $\nnorm{\cdot}_\calF$ a norm on functions from  $\calX$ to $\reals$.
	Let $r: \calX \times \calX$ be a symmetric positive-definite kernel satisfying~\Cref{ass:sym_pd_ker_specturm_decay} and suppose $\sqrt{\lambda_i} \psi_i \in \calF, i \in [d_r(\epsilon / 2)]$, where $\lambda_i, \psi_i$ are the eigenvalues and eigenfunctions of $r$.
	Let $\calV = \sset{\calV_N}_N$ be a family of neural networks satisfying~\Cref{ass:univ_approx_efficiency}.
	Then for any $\epsilon > 0$, there exists a neural network $\phi \in \calV_N$ such that 
	\[\norm{r(x,y) - \iprod{\phi(x)}{\phi(y)}}_{\calF^{\otimes 2}} < \epsilon.\]
	Moreover, the complexity of the neural network required is bounded by
	\[d_r(\epsilon/2) \cdot \calN_\calF\paren{\frac{\epsilon}{4 C(r) d_r(\epsilon/2)}}.\]
\end{theorem}

\begin{proof}
	By Mercer's theorem~\parencite{mercerFunctionsPositive1909, sunMercerTheorem2005, micchelliUniversalKernels2006}, there exists $(\psi_i)_{i \in \mathbb{N}}$, $\lambda_i \geq 0$ such that $r(x,y) = \sum_{i=1}^{\infty}{\lambda_i \psi_i(x) \psi_i(y)}$, where $\psi_i$ and $\lambda_i$ are eigenfunctions and eigenvalues of the integral operator $T_r: L^2(\mathcal{X}) \to L^2(\mathcal{X}),\,T_r: f \mapsto \int_{\mathcal{X}}{r(\cdot, x) f(x) dx}$.
	Furthermore, the convergence of the series is uniform:
	\begin{equation}
		\lim_{n \to \infty} \sup_{x,y \in \mathcal{X}} \lvert r(x,y) - \sum_{i=1}^{n}{\lambda_i \psi_i(x) \psi_i(y) \rvert} = 0.
	\end{equation}
	Using the notation of~\Cref{ass:sym_pd_ker_specturm_decay}, we have that
	\begin{equation}\label{eq:proof_mercer_thm_unif_abs_cv}
		\sup_{x,y \in \mathcal{X}} \left\lvert r(x,y) - \sum_{i=1}^{d_r(\epsilon/2)}{\lambda_i \psi_i(x) \psi_i(y)} \right\rvert < \frac{\epsilon}{2},
	\end{equation}
	where $d_r(\epsilon/2)$ is defined as in~\Cref{ass:sym_pd_ker_specturm_decay}.

	Let $d \coloneqq d_r(\epsilon / 2)$ be the output dimension of the neural network $\phi$ such that it maps from $\calX$ to $\reals^d$. Our approach will be to approximate with the MLP $\phi$ the first $d$ eigenfunctions of the kernel relation $r$, $(\sqrt{\lambda_1} \psi_1, \ldots, \sqrt{\lambda_{d}} \psi_{d})$.
	% The function we would like to approximate with $\phi$ is $(\sqrt{\lambda_1} \psi_1, \ldots, \sqrt{\lambda_{d}} \psi_{d})$, the first $d$ eigenfunctions of the kernel relation $r$.
	By the universal approximation property of $\calV$ (\Cref{ass:univ_approx_efficiency}), for any $\epsilon_1 > 0$, there exists a neural network $\phi = \pparen{\phi_1, \ldots, \phi_d} \in \calV_N$ for some $N$ such that
	\begin{equation}\label{eq:proof_NN_UAP}
		\norm{\phi_i - \sqrt{\lambda_i} \psi_i}_\calF < \epsilon_1, \ \forall i \in \{1, \ldots, d\},
	\end{equation}
	where $\phi_i(x)$ is the $i$-th component of $\phi(x)$. Moreover, by~\Cref{ass:univ_approx_efficiency} there exists such a $\phi$ with a number of neurons bounded by $d \cdot \calN_\calF(\epsilon_1)$.

	Now note that the approximation error for $r$ is bounded by
	\begin{equation}\label{eq:proof_approx_bound}
		\begin{split}
			&\bignorm{r(x,y) - \langle \phi_\theta(x), \phi_\theta(y) \rangle}_{\calF^{\otimes 2}}\\
			&= \norm{ r(x,y) - \sum_{i=1}^{d}{\phi_i(x) \phi_i(y)}}_{\calF^{\otimes 2}} \\
			&\leq \norm{ r(x,y) - \sum_{i=1}^{d}{\lambda_i \psi_i(x) \psi_i(y)}}_{\calF^{\otimes 2}} + \norm{\sum_{i=1}^{d}\lambda_i \psi_i(x) \psi_i(y) - \sum_{i=1}^{d}\phi_i(x) \phi_i(y)}_{\calF^{\otimes 2}}
		\end{split}
	\end{equation}
	The first term is bounded by $\frac{\epsilon}{2}$ by~\eqref{eq:proof_mercer_thm_unif_abs_cv}. The second term can be bounded by noting the for any $x, y \in \calX$
	\begin{equation*}
		\begin{split}
			&\abs{\sum_{i=1}^{d}\lambda_i \psi_i(x) \psi_i(y) - \sum_{i=1}^{d}\phi_i(x) \phi_i(y)} \\
			&\leq \sum_{i=1}^{d}{ \abs{ \lambda_i \psi_i(x) \psi_i(y) - \phi_i(x) \phi_i(y)}} \\
			&\leq \sum_{i=1}^{d}{\paren{
				\abs{\sqrt{\lambda_i} \psi_i(y)} \abs{\sqrt{\lambda_i} \psi_i(y) - \phi_i(y)}
				+ \abs{\sqrt{\lambda_i} \psi_i(y)} \abs{\sqrt{\lambda_i} \psi_i(x) - \phi_i(x)}
				}} \\
			&\leq C(r) \sum_{i=1}^{d}{\paren{
				\abs{\sqrt{\lambda_i} \psi_i(y) - \phi_i(y)}
				+ \abs{\sqrt{\lambda_i} \psi_i(x) - \phi_i(x)}
				}} \\
		\end{split}
	\end{equation*}
	where the last line is by the definition $C(r) \coloneqq \max_{x \in \calX} \lvert \sqrt{\lambda_i} \psi_i(x) \rvert$ (\Cref{ass:sym_pd_ker_specturm_decay}). Now, by~\Cref{eq:proof_NN_UAP}, we have
	\begin{align*}
		&\norm{\sum_{i=1}^{d}\lambda_i \psi_i(x) \psi_i(y) - \sum_{i=1}^{d}\phi_i(x) \phi_i(y)}_{\calF^{\otimes 2}} \\
		&\leq \norm{C(r) \sum_{i=1}^{d}{\paren{\abs{\sqrt{\lambda_i} \psi_i(y) - \phi_i(y)} + \abs{\sqrt{\lambda_i} \psi_i(x) - \phi_i(x)}}}}_{\calF^{\otimes 2}}\\
		&\leq C(r) \sum_{i=1}^{d}\paren{\norm{\sqrt{\lambda_i} \psi_i(y) - \phi_i(y)}_{\calF^{\otimes 2}} + \norm{\sqrt{\lambda_i} \psi_i(x) - \phi_i(x)}_{\calF^{\otimes 2}}}\\
		&\leq 2 C(r) \cdot d \cdot \epsilon_1,
	\end{align*}

	Let the neural network approximation error be $\epsilon_1 = \frac{\epsilon}{4 C(r) d_r(\epsilon / 2)}$ such that the above is bounded by $\epsilon / 2$. 

	Then, by~\eqref{eq:proof_approx_bound}, we have that
	\begin{equation*}
			\bignorm{r(x,y) - \iprod{\phi(x)}{\phi(y)}}_{\calF^{\otimes 2}} \leq \frac{\epsilon}{2} + \frac{\epsilon}{2} = \epsilon.
	\end{equation*}

	Hence, the relation function $r$ is approximated with respect to $\nnorm{\cdot}_{\calF^{\otimes 2}}$ by an inner product of neural networks with number of neurons at most 
	\[d_r(\epsilon / 2) \cdot \calN_\calF\paren{\frac{\epsilon}{4 C(r) d_r(\epsilon / 2)}}.\]

\end{proof}

\Cref{theorem:symmetric_inner_prod_rels_func_class} states that pairwise relations modeled as inner products of neural networks can capture any symmetric positive definite kernel. Moreover, the scale of the neural network needed to achieve a particular approximation error is characterized in terms of the complexity of the kernel relation function and the efficiency of the neural network function class. In particular, this dependence is expressed in the kernel relation's spectrum decay $d_r(\cdot)$ and the neural network's efficiency $\calN_\calF(\cdot)$.

\begin{remark}
	In the results above, the number of neurons is bounded by ``$d_r \cdot \calN$''. This is an upper bound assuming each of the $d_r$ kernel eigenfunctions is modeled independently. In practice, the size of the neural network needed would be smaller since the eigenfunctions can be approximated with a single MLP with a $d_r$-dimensional output, allowing for computations to be re-used across several output dimensions.
\end{remark}

The efficiency of neural networks in approximating arbitrary functions has been studied extensively. Different results in the literature characterize $\calN_{\calF}(\cdot)$ as a function of the class of neural networks (e.g., shallow or deep) and the function class $\calF$ (e.g., degree of smoothness). In the following corollary, we specialize~\Cref{theorem:symmetric_inner_prod_rels_func_class} to approximation with respect to the sup-norm $\nnorm{\cdot}_\calF = \nnorm{\cdot}_\infty$ and $\calF$ the class of Lipschitz functions.

\begin{corollary}\label{cor:sym_iprod_kernel_neuron_bound}
	Consider the setting of~\Cref{theorem:symmetric_inner_prod_rels_func_class}. Suppose further that $\calX = [-1, 1]^d$ and let $L(r, k) \coloneq \max_{i \in [k]} L_i$, where $L_i$ is the Lipschitz constant of $\sqrt{\lambda_i} \psi_i$ and $\lambda_i, \psi_i$ are the eigenvalues and eigenfunctions of the symmetric relation function $r$. Consider $\calV$ to be the class of shallow neural networks with ReLU activations with $\phi \in \calV$ given by
	\begin{equation*}
		\phi(x) = \sum_{k=1}^{N} a_k \mathrm{ReLU}(\iprod{w_k}{x} + b_k),
	\end{equation*}
	where $a_k, b_k \in \reals$, $w_k \in \reals^{d}$ are the parameters, and $N$ is the number of neurons. Then, for any $\epsilon > 0$, there exists $\phi \in \calV_N$ achieving $\sup_{x,y \in \mathcal{X}}{\abs{r(x,y) - \iprod{\phi(x)}{\phi(y)}}} < \epsilon$ with a number of neurons $N$ of the order
	\[\calO\paren{d_r(\epsilon / 2) \cdot \paren{\frac{4 C(r) d_r(\epsilon / 2) L(r, d_r(\epsilon/2))}{\epsilon}}^{d}}.\]
\end{corollary}
The proof of this result follows from~\Cref{theorem:symmetric_inner_prod_rels_func_class} and~\textcite{bachBreakingCurseDimensionality2016}.

%\aanote[margin, noinline]{this discussion is new}
We observe a curse of dimensionality in this approximation result, where the number of neurons needed to achieve a good approximation grows exponentially in the dimension $d$ of the underlying space $\calX$. This phenomenon has been studied in the context of neural network approximation results, and it's well knwon that imposing further structural assumptions on the functions to be approximated can reduce or remove the dependence on dimension. One approach to obtaining a more favorable dependence on the dimension was outlined by~\textcite{barronUniversalApproximation1993}'s seminal work, which identifies a smoothness condition based on the Fourier representation of the underlying function.  The \textit{Barron norm} of a function $f: \reals^d \to \reals$ is defined as
\[\nnorm{f}_{\calB} \coloneq \int_{\reals^d} \nnorm{\hat{\nabla f}(\omega)} d\omega = \int_{\reals^d} \nnorm{\omega} \aabs{\hat{f}(\omega)} d \omega,\]
where $\hat{f}$ denotes the Fourier transform. The advantage of this measure of smoothness is that, for some interesting classes of functions, it scales sub-exponentially with the dimension. In the following corollary, we show that a more favorable dependence on the dimension can be obtained if the eigenfunctions of the relation are smooth in the sense of the Barron norm.

\begin{corollary}\label{cor:sym_iprod_kernel_barron_neuron_bound}
<<<<<<< HEAD
	Consider the setting of~\Cref{theorem:symmetric_inner_prod_rels_func_class}. Suppose that $\calX$ has radius $\mathrm{radius}(\calX)$, so that $\nnorm{x} \leq \mathrm{radius}(\calX),\, \forall x \in \calX$. Assume that $\nnorm{\sqrt{\lambda_i} \psi_i} \leq B(r)$ for $i \in [d_r(\epsilon/2)]$. Consider $\calV$ to be the class of shallow neural networks with sigmoid activations with $\phi \in \calV$ given by
=======
	Consider the setting of~\Cref{theorem:symmetric_inner_prod_rels_func_class}. Suppose that $\calX$ has radius $\mathrm{radius}(\calX)$ (i.e., $\nnorm{x} \leq \mathrm{radius}(\calX),\, \forall x \in \calX$). Assume that $\nnorm{\sqrt{\lambda_i} \psi_i}_{\calB} \leq B(r)$ for $i \in [d_r(\epsilon/2)]$. Consider $\calV$ to be the class of shallow neural networks with sigmoid activations with $\phi \in \calV$ given by
>>>>>>> 8e1150cfcda463a73599492974134e1c2fd29c9b
	\begin{equation*}
		\phi(x) = \sum_{k=1}^{N} a_k \sigma(\iprod{w_k}{x} + b_k),
	\end{equation*}
	where $a_k, b_k \in \reals$, $w_k \in \reals^{d}$ are the parameters, and $N$ is the number of neurons. Then, for any $\epsilon > 0$, there exists $\phi \in \calV_N$ achieving $\nnorm{r(x,y) - \iprod{\phi(x)}{\phi(y)}}_{L^2} < \epsilon$ with a number of neurons $N$ of the order
	\[\calO\paren{d_r(\epsilon / 2)^3 \cdot \frac{\paren{C(r) \cdot \mathrm{radius}(\calX)\, \cdot B(r)}^2}{\epsilon^2}}.\]
\end{corollary}
<<<<<<< HEAD
The proof of this statement follows from~\Cref{theorem:symmetric_inner_prod_rels_func_class} and~\textcite{barronUniversalApproximation1993}.

=======
\begin{proof}
	This follows by~\Cref{theorem:symmetric_inner_prod_rels_func_class} and~\textcite{barronUniversalApproximation1993}.
\end{proof}

This result shows that the size of the neural network needed to approximate $r$ scales with its spectrum decay $d_r(\cdot)$, which can be interpreted as a measure of the dimensionality or ``rank'' of the relation, and the smoothness of its eigenfunctions $B(r)$.
>>>>>>> 8e1150cfcda463a73599492974134e1c2fd29c9b

\section{Function class of asymmetric inner product relations}\label{sec:asymmetric_relations}

In the previous section, we considered symmetric inner product relations where the encoder of the first object is the same as the encoder of the second object. When the underlying relation being modeled is a symmetric `similarity' relation, this is a useful inductive bias. However, in general, relations between objects can be asymmetric. One example of an asymmetric relation is \textit{order} (in fact, anti-symmetric). Such relations cannot be captured by symmetric inner products. In this section, we consider modeling a general (asymmetric) relation $r: \calX \times \calX \to \reals$ as the inner product of two different neural network encodings of a pair of objects,
\begin{equation}\label{eq:asymmetric_iprod_relation}
    r(x, y) = \iprod{\phi(x)}{\psi(y)},
\end{equation}
where $\phi, \psi: \calX \to \reals^d$ are two neural networks.

In this section, we show that inner products of multi-layer perceptrons can approximate any continuous function on $\calX \times \calX$.

We begin with the following very simple lemma which states when the object space $\calX$ is finite, any relation function can be represented as the inner product between two encodings.

\aanote{perhaps this should not be stated as a lemma. or simply removed?}
\begin{lemma}\label{lemma:finite_space_rel}
    Suppose $\calX$ is a finite space. Let $r: \calX \times \calX \to \reals$ be any relation function. Then, there exists $d \leq \abs{\calX}$ and $\phi, \psi: \calX \to \reals^{d}$ such that,
    \begin{equation*}
        r(x, y) = \iprod{\phi(x)}{\psi(y)}, \ \forall x, y \in \calX.
    \end{equation*}
\end{lemma}

\begin{proof}
    Let $x_1, \ldots, x_n$ be an enumeration of $\calX$ where $m = \abs{\calX}$. Let $R \in \reals^{n \times n}$ such that $R_{ij} = r(x_i, x_j)$. There exists many decompositions of the matrix $R$ which would induce valid encodings $\phi, \psi$. One example is rank decomposition. Let $d = \mathrm{rank}(R)$. Then, there exists matrices $P, Q \in \reals^{d \times n}$ such that $R = P^\top Q$. Let $\phi, \psi: \calX \to \reals^{d}$ be defined by
    \begin{equation}
        \phi(x_i) = P_{i, \cdot}, \ \psi(x_i) = Q_{\cdot, i}, \ \forall i \in [m].
    \end{equation}

    Then, $r(x, y) = \iprod{\phi(x)}{\psi(y)}$ for all $x, y \in \calX$.
\end{proof}

Note that if each $x \in \calX$ is a one-hot vector in $\reals^{\abs{\calX}}$, then the result above holds with linear maps $\phi, \psi$. Although very simple, this result has direct implications for domains such as language modeling where $\calX$ is a discrete set of tokens, and hence finite. In such cases,~\Cref{lemma:finite_space_rel} tells us that any relation function can be approximated by inner products of feature maps (i.e., of the form present in the attention mechanisms of Transformers). Moreover, in the case of language, there may be a low-rank structure (e.g., depending on syntax, semantics, etc.) enabling a more modest dimension of the feature maps, $d \ll \abs{\calX}$.

Next, we proceed to show that arbitrary continuous relation functions can be approximated by inner products of two different neural networks. Our strategy will be to first quantize the space $\calX$ then apply the construction above for the finite case.

\begin{theorem}\label{theorem:asymemtric_inner_prod_rel_func_class}
    Suppose the relation function $r: \calX \times \calX \to \reals$ is continuous. In particular, for any $\epsilon > 0$, there exists $\delta(\epsilon) > 0$ such that for any $x, y, x', y'$ satisfying $\norm{x - x'} \leq \delta$, $\norm{y - y'} \leq \delta$, we have $\abs{r(x, y) - r(x', y')} \leq \epsilon$. Then, for any approximation error $\epsilon > 0$ there exists multi-layer perceptrons $\phi, \psi$ such that
    \begin{equation*}
        \abs{r(x,y) - \iprod{\phi(x)}{\psi(y)}} \leq \epsilon, \quad \text{Lebesgue-almost everywhere.}
    \end{equation*}
    Moreover, $\phi, \psi$ can be constructed such that $\phi = L_\phi \circ \eta$, $\psi = L_\psi \circ \eta$ where $\eta$ is a shared 2-layer MLP with $N = \calO(\delta^{- 2 \,\dim(\calX)})$ neurons and $L_\phi, L_\psi$ are linear projections onto $n$-dimensional space, with $n = \calO(\delta^{- \dim(\calX)})$ and $\delta = \delta(\epsilon)$.
\end{theorem}
\begin{proof}
    Let $x_1, \ldots, x_n \in \calX$ be a set of points in $\calX$. Define the Voronoi partition by
    \[V^{(i)} = \sset{x \in \calX \,|\, \norm{x - x_i} \leq \norm{x - x_j} \ \forall j \neq i}.\]
    Let $x_1, \ldots, x_n$ be uniformly distributed in $\calX$ and $n = \calO(\delta(\epsilon)^{- \dim(\calX)})$ so that the maximal diameter of the sets $V^{(1)}, \ldots, V^{(n)}$ is bounded by $\delta(\epsilon)$, $\max_{i \in [n]} \mathrm{diam}(V^{(i)}) \leq \delta$. Let $q: \calX \to \sset{x_1, \ldots, x_n}$ be the quantizer which maps each $x$ to the closest element in $\sset{x_1, \ldots, x_n}$.

    \citet{wuExplicitNeuralNetwork2018} explicitly construct a two-layer neural network $\eta:\calX \to \{0, 1\}^{n}$ such that
    \[{(\eta(x))}_i = 1 \iff x \in V^{(i)}.\]
    The construction contains $n (n - 1)$ neurons in the first layer and $n$ neurons in the second layer, both with the threshold function $\sigma(x) = \bm{1}\sset{x \geq 0}$ as the activation function. Note that sigmoidal activation functions can approximate the step function arbitrarily well. The weights between the first and second layer are sparse. The neural network takes the form,
    \begin{equation*}
        \begin{aligned}
            z_{k,j}^{(1)}(x) &= \sigma\paren{w_{k,j}^{(1)} \cdot x - b_{k,j}^{(1)}}, \quad k,j \in [n], k \neq j\\
            z_k^{(2)}(x) &= \sigma\paren{w_k^{(1)} \cdot \bm{z}^{(1)}(x) - b^{(2)}}, \quad k \in [n] \\
            \eta(x) &= \bm{z}^{(2)}(x) = \paren{z_1^{(2)}(x), \ldots, z_n^{(2)}(x)},
        \end{aligned}
    \end{equation*}
    where the weights are $w_{k,j}^{(1)} = x_k - x_j, b_{k,j}^{(1)} = \frac{1}{2} \iiprod{x_k - x_j}{x_k + x_j}, (w_{k}^{(2)})_{a,b} = \bm{1}\sset{a = k}, b^{(2)} = n - 1$, and $\bm{z}^{(1)} = (z_{k,j}^{(1)})_{k \neq j} \in \reals^{n (n - 1)}$ are the first layer activations.

    Let $R \in \reals^{n \times n}$ be defined by $\bbra{R}_{ij} = r(x_i, x_j)$. The matrix $R$ specifies the relation function on the sample points $x_1, \ldots, x_n$. We have that $R = P^\top Q,\, P,Q \in \reals^{m \times n}$ for some $m \leq n$. 

    Let $\phi = P \circ \eta$ and $\psi = Q \circ \eta$. We will show that the inner product of neural networks $\iprod{\phi(x)}{\psi(y)}$ approximates $r(x, y)$. In fact, this is immediate. Take $x, y \in \calX$. We have
    \begin{align*}
        \abs{r(x, y) - \iprod{\phi(x)}{\psi(y)}} &\leq \abs{r(x, y) - r(q(x), q(y))} + \abs{r(q(x), q(y)) - \iprod{\phi(x)}{\phi(y)}}\\
        &\leq \epsilon + \abs{r(q(x), q(y)) - \eta(x)^\top R\, \eta(y)},
    \end{align*}
    where the second inequality follows by the assumption of continuity on the relation function $r$ and the choice of the quantization diameter $\delta$. Now, note that whenever $x \in \mathrm{int}(V^{(i)})$, we have $\eta(x) = e_i$, where $e_i$ is the canonical basis vector. Hence $r(q(x), q(y)) = \eta(x)^\top R\, \eta(y)$ Lebesgue-almost everywhere. This completes the proof.
\end{proof}

\begin{remark}
    From a learning perspective, we only need to know the value of the relation function at $n$ (uniformly distributed) points $x_1, \ldots, x_n$, with $n = \calO(\delta^{-d})$ depending on the smoothness of $r$ and the dimension of the space $\calX$.
\end{remark}

\begin{remark}
    In the symmetric case, where the relation function $r$ is assumed to be a positive-definite kernel, Mercer's theorem implies an inner product-like structure in $r$, with a ``rank'' determined by the spectral decay of the kernel. In the asymmetric case, the above result assumes only the continuity of the relation function $r$ and no further ``inner-product-like'' structure. In some applications, the relation function may have a ``low-rank'' structure such that it is representable by an inner product between low-dimensional feature maps, enabling more favorable bounds on the size of the neural network needed, even in the asymmetric case.
\end{remark}

\aanote[margin, noinline]{this discussion is new.}
Here, again, we observe the ``curse of dimensionality'' in~\Cref{theorem:asymemtric_inner_prod_rel_func_class}, where the number of neurons needed to obtain a good approximation scales exponentially with the dimension of the underlying space $\calX$. We note that~\Cref{theorem:asymemtric_inner_prod_rel_func_class} makes very mild regularity conditions on the relation function to be approximated---it only needs to be continuous as a function from $\calX \times \calX$ to $\reals$. In general, an exponential dependence on $2\,\cdot\, \dim(\calX)$ is necessary for any method of approximation~\citep{pinkus1999approximation,devore1998nonlinear,maiorov1999lower,maiorov2000near,poggioWhyWhenCan2017}.

This exponential dependence on dimension can be avoided with additional structure on the relation function. Here, we consider a compositional structure where the relation depends on the object vectors only through a smooth low-dimensional feature filter. That is, the target relation function $r: \calX \times \calX \to \reals$ takes the form,
\begin{equation*}
    r(x, y) = \bar{r}(\xi(x), \xi(y)),
\end{equation*}
where $\xi : \calX \to \reals^k$ is a smooth $k$-dimensional feature filter and $\bar{r}$ is a continuous relation on the feature $\xi$. For example, $\calX$ may be a high-dimensional image space, while $\xi$ is a low-dimensional filter representing a particular visual attribute such as color or texture in some patch of the image. We will consider smoothness of $\xi$ as measured by the Barron norm, $\nnorm{f}_{\calB} \coloneq \int_{\reals^d} \nnorm{\hat{\nabla f}(\omega)} d\omega$. We write $\nnorm{\xi}_{\calB}$ to mean $\max_{i \in [k]} \nnorm{\xi_i}_{\calB}$.

\begin{corollary}
    Consider the setting of~\Cref{theorem:asymemtric_inner_prod_rel_func_class}. Suppose the relation function $r: \calX \times \calX \to \reals$ has the form $r(x, y) = \bar{r}(\xi(x), \xi(y))$ where $\xi: \calX \to \reals^k$ has Barron norm $\nnorm{\xi}_\calB$ and $\bar{r}: \reals^k \times \reals^k \to \reals$ is $L$-Lipschitz in the sense that $\aabs{\bar{r}(x, y) - \bar{r}(x', y')} \leq L \cdot (\max(\nnorm{x - x'}_\infty, \nnorm{y - y'}_\infty)), \, \forall x,x',y,y' \in \xi(\calX)$. Then, for any $\epsilon > 0$, there exists $\hat{\xi}, \phi = L_\phi \circ \eta, \psi = L_\psi \circ \eta$, such that
    \[\norm{r(x, y) - \iprod{\phi(\hat{\xi}(x))}{\psi(\hat{\xi}(y))}}_{L^2} \leq \epsilon\]
    where $\hat{\xi}$ is a shallow 1-hidden layer neural network with $\calO(k \cdot \frac{\nnorm{\xi}_{\calB}^2}{\epsilon^2})$ hidden neurons and $k$ output neurons, $\eta$ is a 2-layer neural network with $\calO((\frac{L}{\epsilon})^{2 k})$ neurons, and $L_\phi, L_\psi$ are linear projections onto $\calO((\frac{L}{\epsilon})^{k})$-dimensional space.
\end{corollary}
\begin{proof}
    We have
    \begin{align*}
        &\norm{r(x, y) - \iprod{\phi(\hat{\xi}(x))}{\psi(\hat{\xi}(y))}}_{L^2} \\
        &\leq \norm{\bar{r}(\xi(x), \xi(y)) - \bar{r}(\hat{\xi}(x), \hat{\xi}(y))}_{L^2} + \norm{\bar{r}(\hat{\xi}(x), \hat{\xi}(y)) - \iprod{\phi(\hat{\xi}(x))}{\psi(\hat{\xi}(y))}}_{L^2} \\
        &\leq \norm{L \paren{\infnorm{\xi(x) - \hat{\xi}(x)} + \infnorm{\xi(y) - \hat{\xi}(y)}}}_{L^2} + \norm{\bar{r}(\hat{\xi}(x), \hat{\xi}(y)) - \iprod{\phi(\hat{\xi}(x))}{\psi(\hat{\xi}(y))}}_{L^2}
    \end{align*}
    The first term can be controlled by constructing $\hat{\xi}$ to approximate $\xi$ in the $L^2$ norm according to~\textcite{barronUniversalApproximation1993}'s construction. The number of neurons needed depends on the smoothness of the function $\xi$. The second term can be approximated uniformly (not just in $L^2$) by constructing MLPs $\phi, \psi$ according to~\Cref{theorem:asymemtric_inner_prod_rel_func_class} to approximate $\bar{r}$. Here, the complexity of the neural network class needed depends on $k$ rather than $d$.
\end{proof}

This result shows that if the underlying relation function possesses a compositional structure depending on a low-dimensional smooth feature map, then it can be efficiently approximated by inner products of neural networks. In particular, the number of neurons needed scales with the dimensionality $k$ of the feature filter $\xi$ and the smoothness of $\xi$, as measured by the Barron norm.
\section{Connection to reproducing kernels and RKBS}\label{sec:rkbs_asymmetric_relations}

In~\Cref{sec:symmetric_relations} we showed that the function class of symmetric inner products of neural networks is the set of symmetric positive-definite kernels---that is, reproducing kernels of reproducing kernel Hilbert spaces (RKHS). There exists a similar interpretation of the function class of asymmetric inner products of neural networks in terms of the reproducing kernels of \textit{reproducing kernel Banach spaces} (RKBS).

Recall that a reproducing kernel Hilbert space $\calH$ is a Hilbert space of functions on a space $\calX$ in which the point evaluation functionals $f \mapsto f(x)$ are continuous. There is a one-to-one identification between RKHSs and symmetric positive definite kernels $K: \calX \times \calX \to \reals$ such that $\iprod{K(x, \cdot)}{f}_\calH = f(x)$ \parencite{aronszajn1950theory}.~\parencite{mercerFunctionsPositive1909} further shows that an RKHS can be identified with a feature map via the spectral decomposition of the integral operator $T_K: L_2(\calX) \to L_2(\calX)$ defined by $T_K f(x) = \int_\calX K(x, y) f(y) dy$. Every feature map $\phi: \calX \to \calW$ defines a symmetric positive definite kernel $K(x, y) = \iprod{\phi(x)}{\phi(y)}_\calW$ (hence, an RKHS) and every symmetric positive definite kernel has infinitely many feature map representations.

\Cref{theorem:symmetric_inner_prod_rels_func_class} shows that the function class of symmetric inner products of neural networks is the set of reproducing kernels of RKHS function spaces.A reproducing kernel Hilbert space is, as the name suggests, a \textit{Hilbert space} of functions on some space, $\calX$. The linear structure of a Hilbert space makes the kinds of geometries it can capture relatively restrictive. In particular, any two Hilbert spaces with the same dimension are isometrically isomorphic. Banach spaces, which have fewer structural assumptions, can capture richer geometric structures. Hence, a reproducing kernel Banach space can capture richer geometries between functions than an RKHS. In particular, in contrast to an RKHS, the reproducing kernel of an RKBS need not be symmetric or positive definite. In this section, we show that the function class of asymmetric inner products of neural networks has an interpretation in terms of the reproducing kernels of RKBSs, mirroring the result for symmetric inner products of neural networks
\footnote{While working on extending~\Cref{theorem:symmetric_inner_prod_rels_func_class} to the asymmetric case, we came across an interesting paper from 1919 which studied non-symmetric kernels of positive type~\parencite{seelyNonSymmetricKernels1919}. Mercer's work from a decade earlier studied \textit{symmetric} kernels of positive type---the proof of~\Cref{theorem:symmetric_inner_prod_rels_func_class} relied on the property of uniform convergence of the series of characteristic functions: $\sum_i \lambda_i \psi_i(s)\psi_i(t) \to K(s,t)$. Ultimately, we didn't use the results of~\parencite{seelyNonSymmetricKernels1919} since asymmetric neural network inner products need not be of positive-type (the function class is in fact larger).}.

\subsection{Background on reproducing kernel Banach spaces}

\begin{definition}[Reproducing Kernel Banach Space]
    A \textbf{reproducing kernel Banach space} on a space $\calX$ is a Banach space $\calB$ of functions on $\calX$, satisfying:
    \begin{enumerate}
        \item $\calB$ is \textit{reflexive}. That is, $(\calB^*)^* = \calB$, where $\calB^*$ is the dual space of $\calB$. Furthermore, $\calB^*$ is isometric to a Banach space $\calB^\#$ of functions on $\calX$.
        \item The point evaluation functionals $f \mapsto f(x)$ are continuous on both $\calB$ and $\calB^\#$.
    \end{enumerate}
\end{definition}

This definition is a strict generalization of reproducing kernel Hilbert spaces, as any RKHS $\calH$ on $\calX$ is also an RKBS ((1) is implied by the Riesz representation theorem). While the identification $\calB^\#$ is not unique, we can choose some identification arbitrarily and denote it by $\calB^*$ for ease of notation (by assumption, all identifications are isometric to each other). Thus, if $\calB$ is an RKBS, $\calB^*$ is also an RKBS.

Similar to an RKHS, an RKBS also has a \textit{reproducing kernel}. To state the result, for a normed vector space $\calV$ and its dual space $\calV^*$, we define the bilinear form
\begin{equation}\label{eq:bilinear_form}
    \begin{split}
        \calV \times \calV^* &\to \reals\\
        (u, v^*)_\calV &\mapsto v^*(u).
    \end{split}
\end{equation}

\parencite[Theorem 2]{zhangReproducingKernel2009} shows that for any RKBS $\calB$ there exists a unique reproducing kernel $K: \calX \times \calX \to \bbC$ which recovers point evaluations,
\begin{align}
    f(x) &= \paren{f, K(\cdot, x)}_\calB, \forall f \in \calB, \\
    f^*(x) &= \paren{K(x, \cdot), f^*}_\calB \forall f^* \in \calB^*,
\end{align}

and such that the span of $K(x, \cdot)$ is dense in $\calB$ and the span of $K(\cdot, x)$ is dense in $\calB^*$,
\begin{align}
    \overline{\text{span}}\{K(x, \cdot): x \in \calX\} &= \calB, \\
    \overline{\text{span}}\{K(\cdot, x): x \in \calX\} &= \calB^*.
\end{align}

Finally,

\begin{equation}
    K(x, y) = \paren{K(x, \cdot), K(\cdot, y)}_\calB, \ \forall x, y \in \calX.
\end{equation}

Unlike RKHSs, while each RKBS has a unique reproducing kernel, different RKBSs may have the same reproducing kernels.

Furthermore,~\parencite[Theorems 3 and 4]{zhangReproducingKernel2009} show that a kernel $K: \calX \times \calX \to \bbC$ is the reproducing kernel of some RKBS if and only if it has a feature map representation. Crucially for us, the feature map representation is more versatile than the one for RKHSs. Let $\calW$ be a reflexive Banach space with dual space $\calW^*$. Consider a pair of feature maps $\Phi$ and $\Phi^*$, mapping to each feature space, respectively. That is,
\begin{equation*}
    \Phi: \calX \to \calW, \ \Phi^*: \calX \to \calW^*,
\end{equation*}
where we call $\Phi, \Phi^*$ the \textit{pair} of feature maps and $\calW, \calW^*$ the pair of feature spaces. Suppose that the span of the image of the feature maps under $\calX$ is dense in their respective feature spaces. That is,
\begin{equation}
    \overline{\text{span}}\{\Phi(x): x \in \calX\} = \calW, \ \overline{\text{span}}\{\Phi^*(x): x \in \calX\} = \calW^*.
\end{equation}

Then, by~\parencite[Theorem 3]{zhangReproducingKernel2009}, the feature maps $\Phi, \Phi^*$ induce an RKBS defined by
\begin{align}
    \calB &:= \set{f_w: x \mapsto (\Phi^*(x))(w), w \in \calW} \\
    \norm{f_w}_\calB &:= \norm{w}_\calW,
\end{align}
with the dual space $\calB^*$ defined by
\begin{align}
    \calB^* &:= \set{f_{w^*}: x \mapsto w^*(\Phi(x)), w^* \in \calW^*} \\
    \norm{f_{w^*}}_{\calB^*} &:= \norm{w^*}_{\calW^*}.
\end{align}

Furthermore, for any RKBS, there exists some feature spaces $\calW, \calW^*$ and feature maps $\Phi, \Phi^*$ such that the above construction yields that RKBS~\parencite[Theorem 4]{zhangReproducingKernel2009}.

\subsection{Asymmetric inner products of neural networks model kernels of reproducing kernel Banach spaces}

Observe that for a RKBS with feature-map representation given by $\Phi, \Phi^*$, it's reproducing kernel is given by
\begin{equation}
    K(x, y) = \paren{\Phi(x), \Phi^*(y)}_{\calW}, \ x, y \in \calX,
\end{equation}
where $\paren{\cdot, \cdot}_{\calW}$ is the bilinear form on $\calW$ defined in~\Cref{eq:bilinear_form}.

This form is reminiscent of the asymmetric inner product of neural networks,
\begin{equation}
    r(x, y) = \iprod{\phi(x)}{\psi(y)}, \ x, y \in \calX,
\end{equation}
where $\phi, \psi: \calX \to \reals^{d}$ is a pair of learned feature maps. The following theorem states that when $\calB$ is an RKBS on $\calX$ with a feature map representation whose feature space $\calW$ is a Hilbert space, its reproducing kernel can be approximated by an asymmetric inner product of neural networks. The proof is similar to~\Cref{theoorem:asymemtric_inner_prod_rel_func_class}.

\begin{theorem}\label{thm:asymmetric_inner_prod_approximates_rkbs}
   Suppose $\calX$ is a compact metric space. Suppose $r: \calX \times \calX \to \reals$ is the reproducing kernel of some RKBS $\calB$ on $\calX$ which admits a feature map representation with a feature space $\calW$ which is a Hilbert space. Consider the model,
   \begin{equation}
    \hat{r}(x, y) = \iprod{\phi_\theta(x)}{\psi_\theta(y)},
   \end{equation}
    where $\phi, \psi: \calX \to \reals^{d}$ are multi-layer perceptrons. Then, for any $\varepsilon > 0$, there exists multi-layer perceptrons with parameters $\theta$ such that
   \begin{equation*}
        \sup_{x,y \in \calX} \abs{r(x, y) - \hat{r}(x, y)} \leq \varepsilon
   \end{equation*}
\end{theorem}

\begin{proof}
    \hphantom{~}

    By assumption, there exists a Hilbert space $\calW$ and a pair of feature maps $\Phi: \calX \to \calW, \Phi^*: \calX \to \calW$ such that,
    \begin{equation*}
        r(x, y) = \paren{\Phi(x), \Phi^*(y)}_{\calW} \equiv (\Phi^*(y))(x), \ x, y \in \calX.
    \end{equation*}

    Without loss of generality, we can restrict our attention to the feature space $\calW = \ell^2(\bbN)$, since any two Hilbert spaces with equal dimension are isometrically isomorphic. The dual space is $\calW^* = \ell^2(\bbN)$. Hence, for feature maps $\Phi, \Phi^*$, the ground truth relation to be approximated is,
    \begin{equation*}
        r(x, y) = \paren{\Phi(x), \Phi^*(y)}_{\ell^2(\bbN)} \equiv (\Phi^*(y))(\Phi(x)), \ x, y \in \calX.
    \end{equation*}

    By the Riesz representation theorem~\parencite{riesz_citation}, there exists a unique element in $u_{\Phi^*(y)} \in \ell^2(\bbN)$ such that,
    \begin{equation*}
        (\Phi^*(y))(w) = \iprod{w}{u_{\Phi^*(y)}}_{\ell^2(\bbN)}, \ \forall w \in \ell^2(\bbN).
    \end{equation*}

    Let $\sigma: \ell^2(\bbN)^* \to \ell^2(\bbN)$ denote the mapping from an element in the dual space to its Riesz representation. $\sigma$ is a bijective isometric antilinear isomorphism. (E.g., the Riesz representation can be constructed via an orthonormal basis through $\sigma(w^*) = \sum_{i \in I} w^{*}(e_i) e_i$, where $\set{e_i}_{i \in I}$ is some basis for $\calW$.)

    Thus, the relation function on $\calX \times \calX$ that we need to approximate is,
    \begin{equation*}
        r(x, y) = \iprod{\sigma \circ \Phi^* (y)}{\Phi(x)}_{\calW}, \ x, y \in \calX.
    \end{equation*}

    We do this by approximating $\Phi: \calX \to \calW$ with the MLP $\psi$ and approximating $\sigma \circ \Phi^*: \calX \to \calW$ with the MLP $\phi$.

    First, since $\Phi(x), \sigma \circ \Phi^*(y) \in \ell^2(\bbN), \forall x, y$, and $\calX$ is compact, we have
    \begin{equation*}
        \lim_{n \to \infty} \sup_{x,y \in \calX} \abs{r(x, y) - \sum_{i=1}^{n} (\Phi(x))_i \cdot (\sigma(\Phi^*(y)))_i} = 0.
    \end{equation*}

    Thus, let $d$ be such that,
    \begin{equation}\label{eq:thm1_proof_eq1}
        \sup_{x,y \in \calX} \abs{r(x, y) - \sum_{i=1}^{d} (\Phi(x))_i \cdot (\sigma(\Phi^*(y)))_i} < \frac{\varepsilon}{2}.
    \end{equation}

    Now, let the MLPs $\phi_\theta, \psi_\theta$ be functions from $\calX$ to $\reals^{d}$ (i.e., we specify the architecture of the MLPs such that the output space is $d$-dimensional). Let $\paren{(\Phi(x))_1, \ldots, (\Phi(x))_{d}}$ be the function to be approximated by the MLP $\phi_\theta$ and let $\paren{(\sigma(\Phi^*(y)))_1, \ldots, (\sigma(\Phi^*(y)))_{d}}$ be the function to be approximated by the MLP $\psi_\theta$. By universal approximation results on MLPs (e.g.,~\parencite{barronUniversalApproximation1993, cybenkoApproximationSuperpositions1989, hornikMultilayerFeedforward1989}), for any $\tilde{\varepsilon} > 0$, there exists parameters $\theta$ such that,
    \begin{equation}\label{eq:thm1_proof_eq2}
        \sup_{x \in \calX} \abs{(\phi_\theta(x))_i - (\Phi(x))_{i}} < \tilde{\varepsilon} \ \text{ and } \ \sup_{x \in \calX} \abs{(\psi_\theta(y))_i - (\sigma(\Phi^*(x)))_{i}} < \tilde{\varepsilon}, \ \forall i \in \set{1, \ldots, d}.
    \end{equation}

    Now,
    \begin{align*}
        &\sup_{x,y \in \calX} \abs{r(x,y) - \hat{r}(x,y)} \\
        &= \sup_{x,y \in \calX} \abs{r(x,y) - \iprod{\phi_\theta(x)}{\psi_\theta(y)}} \\
        &\leq \sup_{x,y\in \calX} \paren{\abs{r(x,y) - \sum_{i=1}^{d} (\Phi(x))_i \cdot (\sigma(\Phi^*(y)))_i} + \abs{\sum_{i=1}^{d} (\Phi(x))_i \cdot (\sigma(\Phi^*(y)))_i - \iprod{\phi_\theta(x)}{\psi_\theta(y)}}} \\
    \end{align*}

    The first term is less than $\varepsilon / 2$ by~\Cref{eq:thm1_proof_eq1}. Now, we bound the second term uniformly on $x,y \in \calX$,
    \begin{align*}
        &\abs{\sum_{i=1}^{d} (\Phi(x))_i \cdot (\sigma(\Phi^*(y)))_i - \iprod{\phi_\theta(x)}{\psi_\theta(y)}} \\
        &\leq \sum_{i=1}^{d} \abs{(\Phi(x))_i \cdot (\sigma(\Phi^*(y)))_i - (\phi_\theta(x))_i(\psi_i(y))_i} \\
        &\leq \sum_{i=1}^{d} \abs{(\phi_\theta(x))_i} \abs{(\sigma(\Phi^*(y)))_i - (\psi_i(y))_i} + \abs{(\psi_\theta(y))_i} \abs{(\Phi(x))_i - (\phi_\theta(x))_i}\\
    \end{align*}

    Let $\tilde{\varepsilon}$ in~\Cref{eq:thm1_proof_eq2} be small enough such that the above is smaller than $\varepsilon / 2$. This shows that
    \begin{equation*}
        \sup_{x,y \in \calX} \abs{r(x,y) - \tilde{r}_i(x,y)} \leq \frac{\varepsilon}{2} + \frac{\varepsilon}{2} = \varepsilon.
    \end{equation*}
\end{proof}

\begin{remark}
    The reason we assume that the underlying RKBS $\calB$ admits a feature map representation with feature space $\calW$ which is a Hilbert space is so that we can use the Riesz representation theorem. The Riesz representation theorem is what links the broad framework of reproducing kernel Banach spaces back to the inductive bias of modeling relations as inner products of feature maps.
\end{remark}

\begin{remark}
    In~\parencite{zhangReproducingKernel2009}, the authors explore a specialization of reproducing kernel Banach spaces in which $\calB$ has a semi-inner product. This added structure grants semi-inner product RKBSs some desirable properties which RKHSs have but general RKBSs lack (e.g., convergence in the space implies pointwise convergence, weak universality of kernels, etc.). However, their notion of a semi-inner product is too restrictive to allow for our model $\iprod{\phi(x)}{\psi(x)}$.
\end{remark}
\section{Discussion}

The analysis in this note underscores the importance of kernels for learning relations and attention mechanisms. In the symmetric case, the assumption of a positive-definite kernel function is natural, leading to the standard framework of reproducing kernel Hilbert spaces. In the asymmetric case, which is arguably more important and applicable for relational learning, a different technical approach is needed, and reproducing kernel Banach spaces arises naturally. After completing the work presented here, we became aware of the related work of \citet{wright2021transformers}, which makes this connection as well.

The results presented here can be extended in several ways. For example, the bounds on the number of neurons in a perceptron that suffice to approximate a relation function to a given accuracy can likely be sharpened, drawing on the extensive literature on approximation properties of neural networks \citep[e.g.,][]{petrushev1998approximation,pinkus1999approximation,makovoz1998uniform,burger2001error,maiorov2006approximation,bachBreakingCurseDimensionality2016}. In terms of attention mechanisms in transformers, our initial focus was on approximating the most relevant key to a given query. The representation theorem of \citet{debreuRepresentationPreferenceOrdering1954} is used to express the problem in terms of a utility function, which is then approximated. It would be of interest to derive approximation bounds for the full distribution of attention values that are computed by the softmax function in Transformers. Finally, when considering relational learning, the possibility of higher-order, recursive relations, naturally arises \citep[e.g.,][]{altabaaRelationalConvolutionalNetworks2023}, and it may be interesting to study function spaces of hierarchical relations in such settings.


\printbibliography

\end{document}